
% Default to the notebook output style

    


% Inherit from the specified cell style.




    
\documentclass[11pt]{article}

    
    
    \usepackage[T1]{fontenc}
    % Nicer default font than Computer Modern for most use cases
    \usepackage{palatino}

    % Basic figure setup, for now with no caption control since it's done
    % automatically by Pandoc (which extracts ![](path) syntax from Markdown).
    \usepackage{graphicx}
    % We will generate all images so they have a width \maxwidth. This means
    % that they will get their normal width if they fit onto the page, but
    % are scaled down if they would overflow the margins.
    \makeatletter
    \def\maxwidth{\ifdim\Gin@nat@width>\linewidth\linewidth
    \else\Gin@nat@width\fi}
    \makeatother
    \let\Oldincludegraphics\includegraphics
    % Set max figure width to be 80% of text width, for now hardcoded.
    \renewcommand{\includegraphics}[1]{\Oldincludegraphics[width=.8\maxwidth]{#1}}
    % Ensure that by default, figures have no caption (until we provide a
    % proper Figure object with a Caption API and a way to capture that
    % in the conversion process - todo).
    \usepackage{caption}
    \DeclareCaptionLabelFormat{nolabel}{}
    \captionsetup{labelformat=nolabel}

    \usepackage{adjustbox} % Used to constrain images to a maximum size 
    \usepackage{xcolor} % Allow colors to be defined
    \usepackage{enumerate} % Needed for markdown enumerations to work
    \usepackage{geometry} % Used to adjust the document margins
    \usepackage{amsmath} % Equations
    \usepackage{amssymb} % Equations
    \usepackage{textcomp} % defines textquotesingle
    % Hack from http://tex.stackexchange.com/a/47451/13684:
    \AtBeginDocument{%
        \def\PYZsq{\textquotesingle}% Upright quotes in Pygmentized code
    }
    \usepackage{upquote} % Upright quotes for verbatim code
    \usepackage{eurosym} % defines \euro
    \usepackage[mathletters]{ucs} % Extended unicode (utf-8) support
    \usepackage[utf8x]{inputenc} % Allow utf-8 characters in the tex document
    \usepackage{fancyvrb} % verbatim replacement that allows latex
    \usepackage{grffile} % extends the file name processing of package graphics 
                         % to support a larger range 
    % The hyperref package gives us a pdf with properly built
    % internal navigation ('pdf bookmarks' for the table of contents,
    % internal cross-reference links, web links for URLs, etc.)
    \usepackage{hyperref}
    \usepackage{longtable} % longtable support required by pandoc >1.10
    \usepackage{booktabs}  % table support for pandoc > 1.12.2
    \usepackage[normalem]{ulem} % ulem is needed to support strikethroughs (\sout)
                                % normalem makes italics be italics, not underlines
    

    
    
    % Colors for the hyperref package
    \definecolor{urlcolor}{rgb}{0,.145,.698}
    \definecolor{linkcolor}{rgb}{.71,0.21,0.01}
    \definecolor{citecolor}{rgb}{.12,.54,.11}

    % ANSI colors
    \definecolor{ansi-black}{HTML}{3E424D}
    \definecolor{ansi-black-intense}{HTML}{282C36}
    \definecolor{ansi-red}{HTML}{E75C58}
    \definecolor{ansi-red-intense}{HTML}{B22B31}
    \definecolor{ansi-green}{HTML}{00A250}
    \definecolor{ansi-green-intense}{HTML}{007427}
    \definecolor{ansi-yellow}{HTML}{DDB62B}
    \definecolor{ansi-yellow-intense}{HTML}{B27D12}
    \definecolor{ansi-blue}{HTML}{208FFB}
    \definecolor{ansi-blue-intense}{HTML}{0065CA}
    \definecolor{ansi-magenta}{HTML}{D160C4}
    \definecolor{ansi-magenta-intense}{HTML}{A03196}
    \definecolor{ansi-cyan}{HTML}{60C6C8}
    \definecolor{ansi-cyan-intense}{HTML}{258F8F}
    \definecolor{ansi-white}{HTML}{C5C1B4}
    \definecolor{ansi-white-intense}{HTML}{A1A6B2}

    % commands and environments needed by pandoc snippets
    % extracted from the output of `pandoc -s`
    \providecommand{\tightlist}{%
      \setlength{\itemsep}{0pt}\setlength{\parskip}{0pt}}
    \DefineVerbatimEnvironment{Highlighting}{Verbatim}{commandchars=\\\{\}}
    % Add ',fontsize=\small' for more characters per line
    \newenvironment{Shaded}{}{}
    \newcommand{\KeywordTok}[1]{\textcolor[rgb]{0.00,0.44,0.13}{\textbf{{#1}}}}
    \newcommand{\DataTypeTok}[1]{\textcolor[rgb]{0.56,0.13,0.00}{{#1}}}
    \newcommand{\DecValTok}[1]{\textcolor[rgb]{0.25,0.63,0.44}{{#1}}}
    \newcommand{\BaseNTok}[1]{\textcolor[rgb]{0.25,0.63,0.44}{{#1}}}
    \newcommand{\FloatTok}[1]{\textcolor[rgb]{0.25,0.63,0.44}{{#1}}}
    \newcommand{\CharTok}[1]{\textcolor[rgb]{0.25,0.44,0.63}{{#1}}}
    \newcommand{\StringTok}[1]{\textcolor[rgb]{0.25,0.44,0.63}{{#1}}}
    \newcommand{\CommentTok}[1]{\textcolor[rgb]{0.38,0.63,0.69}{\textit{{#1}}}}
    \newcommand{\OtherTok}[1]{\textcolor[rgb]{0.00,0.44,0.13}{{#1}}}
    \newcommand{\AlertTok}[1]{\textcolor[rgb]{1.00,0.00,0.00}{\textbf{{#1}}}}
    \newcommand{\FunctionTok}[1]{\textcolor[rgb]{0.02,0.16,0.49}{{#1}}}
    \newcommand{\RegionMarkerTok}[1]{{#1}}
    \newcommand{\ErrorTok}[1]{\textcolor[rgb]{1.00,0.00,0.00}{\textbf{{#1}}}}
    \newcommand{\NormalTok}[1]{{#1}}
    
    % Additional commands for more recent versions of Pandoc
    \newcommand{\ConstantTok}[1]{\textcolor[rgb]{0.53,0.00,0.00}{{#1}}}
    \newcommand{\SpecialCharTok}[1]{\textcolor[rgb]{0.25,0.44,0.63}{{#1}}}
    \newcommand{\VerbatimStringTok}[1]{\textcolor[rgb]{0.25,0.44,0.63}{{#1}}}
    \newcommand{\SpecialStringTok}[1]{\textcolor[rgb]{0.73,0.40,0.53}{{#1}}}
    \newcommand{\ImportTok}[1]{{#1}}
    \newcommand{\DocumentationTok}[1]{\textcolor[rgb]{0.73,0.13,0.13}{\textit{{#1}}}}
    \newcommand{\AnnotationTok}[1]{\textcolor[rgb]{0.38,0.63,0.69}{\textbf{\textit{{#1}}}}}
    \newcommand{\CommentVarTok}[1]{\textcolor[rgb]{0.38,0.63,0.69}{\textbf{\textit{{#1}}}}}
    \newcommand{\VariableTok}[1]{\textcolor[rgb]{0.10,0.09,0.49}{{#1}}}
    \newcommand{\ControlFlowTok}[1]{\textcolor[rgb]{0.00,0.44,0.13}{\textbf{{#1}}}}
    \newcommand{\OperatorTok}[1]{\textcolor[rgb]{0.40,0.40,0.40}{{#1}}}
    \newcommand{\BuiltInTok}[1]{{#1}}
    \newcommand{\ExtensionTok}[1]{{#1}}
    \newcommand{\PreprocessorTok}[1]{\textcolor[rgb]{0.74,0.48,0.00}{{#1}}}
    \newcommand{\AttributeTok}[1]{\textcolor[rgb]{0.49,0.56,0.16}{{#1}}}
    \newcommand{\InformationTok}[1]{\textcolor[rgb]{0.38,0.63,0.69}{\textbf{\textit{{#1}}}}}
    \newcommand{\WarningTok}[1]{\textcolor[rgb]{0.38,0.63,0.69}{\textbf{\textit{{#1}}}}}
    
    
    % Define a nice break command that doesn't care if a line doesn't already
    % exist.
    \def\br{\hspace*{\fill} \\* }
    % Math Jax compatability definitions
    \def\gt{>}
    \def\lt{<}
    % Document parameters
    \title{Microwave}
    
    
    

    % Pygments definitions
    
\makeatletter
\def\PY@reset{\let\PY@it=\relax \let\PY@bf=\relax%
    \let\PY@ul=\relax \let\PY@tc=\relax%
    \let\PY@bc=\relax \let\PY@ff=\relax}
\def\PY@tok#1{\csname PY@tok@#1\endcsname}
\def\PY@toks#1+{\ifx\relax#1\empty\else%
    \PY@tok{#1}\expandafter\PY@toks\fi}
\def\PY@do#1{\PY@bc{\PY@tc{\PY@ul{%
    \PY@it{\PY@bf{\PY@ff{#1}}}}}}}
\def\PY#1#2{\PY@reset\PY@toks#1+\relax+\PY@do{#2}}

\expandafter\def\csname PY@tok@cpf\endcsname{\let\PY@it=\textit\def\PY@tc##1{\textcolor[rgb]{0.25,0.50,0.50}{##1}}}
\expandafter\def\csname PY@tok@kc\endcsname{\let\PY@bf=\textbf\def\PY@tc##1{\textcolor[rgb]{0.00,0.50,0.00}{##1}}}
\expandafter\def\csname PY@tok@sc\endcsname{\def\PY@tc##1{\textcolor[rgb]{0.73,0.13,0.13}{##1}}}
\expandafter\def\csname PY@tok@gi\endcsname{\def\PY@tc##1{\textcolor[rgb]{0.00,0.63,0.00}{##1}}}
\expandafter\def\csname PY@tok@mo\endcsname{\def\PY@tc##1{\textcolor[rgb]{0.40,0.40,0.40}{##1}}}
\expandafter\def\csname PY@tok@s\endcsname{\def\PY@tc##1{\textcolor[rgb]{0.73,0.13,0.13}{##1}}}
\expandafter\def\csname PY@tok@na\endcsname{\def\PY@tc##1{\textcolor[rgb]{0.49,0.56,0.16}{##1}}}
\expandafter\def\csname PY@tok@gd\endcsname{\def\PY@tc##1{\textcolor[rgb]{0.63,0.00,0.00}{##1}}}
\expandafter\def\csname PY@tok@o\endcsname{\def\PY@tc##1{\textcolor[rgb]{0.40,0.40,0.40}{##1}}}
\expandafter\def\csname PY@tok@nn\endcsname{\let\PY@bf=\textbf\def\PY@tc##1{\textcolor[rgb]{0.00,0.00,1.00}{##1}}}
\expandafter\def\csname PY@tok@kd\endcsname{\let\PY@bf=\textbf\def\PY@tc##1{\textcolor[rgb]{0.00,0.50,0.00}{##1}}}
\expandafter\def\csname PY@tok@sr\endcsname{\def\PY@tc##1{\textcolor[rgb]{0.73,0.40,0.53}{##1}}}
\expandafter\def\csname PY@tok@cp\endcsname{\def\PY@tc##1{\textcolor[rgb]{0.74,0.48,0.00}{##1}}}
\expandafter\def\csname PY@tok@nv\endcsname{\def\PY@tc##1{\textcolor[rgb]{0.10,0.09,0.49}{##1}}}
\expandafter\def\csname PY@tok@k\endcsname{\let\PY@bf=\textbf\def\PY@tc##1{\textcolor[rgb]{0.00,0.50,0.00}{##1}}}
\expandafter\def\csname PY@tok@mb\endcsname{\def\PY@tc##1{\textcolor[rgb]{0.40,0.40,0.40}{##1}}}
\expandafter\def\csname PY@tok@m\endcsname{\def\PY@tc##1{\textcolor[rgb]{0.40,0.40,0.40}{##1}}}
\expandafter\def\csname PY@tok@gt\endcsname{\def\PY@tc##1{\textcolor[rgb]{0.00,0.27,0.87}{##1}}}
\expandafter\def\csname PY@tok@err\endcsname{\def\PY@bc##1{\setlength{\fboxsep}{0pt}\fcolorbox[rgb]{1.00,0.00,0.00}{1,1,1}{\strut ##1}}}
\expandafter\def\csname PY@tok@ch\endcsname{\let\PY@it=\textit\def\PY@tc##1{\textcolor[rgb]{0.25,0.50,0.50}{##1}}}
\expandafter\def\csname PY@tok@ss\endcsname{\def\PY@tc##1{\textcolor[rgb]{0.10,0.09,0.49}{##1}}}
\expandafter\def\csname PY@tok@vg\endcsname{\def\PY@tc##1{\textcolor[rgb]{0.10,0.09,0.49}{##1}}}
\expandafter\def\csname PY@tok@gr\endcsname{\def\PY@tc##1{\textcolor[rgb]{1.00,0.00,0.00}{##1}}}
\expandafter\def\csname PY@tok@gu\endcsname{\let\PY@bf=\textbf\def\PY@tc##1{\textcolor[rgb]{0.50,0.00,0.50}{##1}}}
\expandafter\def\csname PY@tok@cm\endcsname{\let\PY@it=\textit\def\PY@tc##1{\textcolor[rgb]{0.25,0.50,0.50}{##1}}}
\expandafter\def\csname PY@tok@nd\endcsname{\def\PY@tc##1{\textcolor[rgb]{0.67,0.13,1.00}{##1}}}
\expandafter\def\csname PY@tok@ge\endcsname{\let\PY@it=\textit}
\expandafter\def\csname PY@tok@bp\endcsname{\def\PY@tc##1{\textcolor[rgb]{0.00,0.50,0.00}{##1}}}
\expandafter\def\csname PY@tok@kp\endcsname{\def\PY@tc##1{\textcolor[rgb]{0.00,0.50,0.00}{##1}}}
\expandafter\def\csname PY@tok@c\endcsname{\let\PY@it=\textit\def\PY@tc##1{\textcolor[rgb]{0.25,0.50,0.50}{##1}}}
\expandafter\def\csname PY@tok@sx\endcsname{\def\PY@tc##1{\textcolor[rgb]{0.00,0.50,0.00}{##1}}}
\expandafter\def\csname PY@tok@vc\endcsname{\def\PY@tc##1{\textcolor[rgb]{0.10,0.09,0.49}{##1}}}
\expandafter\def\csname PY@tok@se\endcsname{\let\PY@bf=\textbf\def\PY@tc##1{\textcolor[rgb]{0.73,0.40,0.13}{##1}}}
\expandafter\def\csname PY@tok@sb\endcsname{\def\PY@tc##1{\textcolor[rgb]{0.73,0.13,0.13}{##1}}}
\expandafter\def\csname PY@tok@mf\endcsname{\def\PY@tc##1{\textcolor[rgb]{0.40,0.40,0.40}{##1}}}
\expandafter\def\csname PY@tok@nt\endcsname{\let\PY@bf=\textbf\def\PY@tc##1{\textcolor[rgb]{0.00,0.50,0.00}{##1}}}
\expandafter\def\csname PY@tok@kr\endcsname{\let\PY@bf=\textbf\def\PY@tc##1{\textcolor[rgb]{0.00,0.50,0.00}{##1}}}
\expandafter\def\csname PY@tok@mi\endcsname{\def\PY@tc##1{\textcolor[rgb]{0.40,0.40,0.40}{##1}}}
\expandafter\def\csname PY@tok@go\endcsname{\def\PY@tc##1{\textcolor[rgb]{0.53,0.53,0.53}{##1}}}
\expandafter\def\csname PY@tok@si\endcsname{\let\PY@bf=\textbf\def\PY@tc##1{\textcolor[rgb]{0.73,0.40,0.53}{##1}}}
\expandafter\def\csname PY@tok@s1\endcsname{\def\PY@tc##1{\textcolor[rgb]{0.73,0.13,0.13}{##1}}}
\expandafter\def\csname PY@tok@sh\endcsname{\def\PY@tc##1{\textcolor[rgb]{0.73,0.13,0.13}{##1}}}
\expandafter\def\csname PY@tok@kn\endcsname{\let\PY@bf=\textbf\def\PY@tc##1{\textcolor[rgb]{0.00,0.50,0.00}{##1}}}
\expandafter\def\csname PY@tok@cs\endcsname{\let\PY@it=\textit\def\PY@tc##1{\textcolor[rgb]{0.25,0.50,0.50}{##1}}}
\expandafter\def\csname PY@tok@ni\endcsname{\let\PY@bf=\textbf\def\PY@tc##1{\textcolor[rgb]{0.60,0.60,0.60}{##1}}}
\expandafter\def\csname PY@tok@c1\endcsname{\let\PY@it=\textit\def\PY@tc##1{\textcolor[rgb]{0.25,0.50,0.50}{##1}}}
\expandafter\def\csname PY@tok@ne\endcsname{\let\PY@bf=\textbf\def\PY@tc##1{\textcolor[rgb]{0.82,0.25,0.23}{##1}}}
\expandafter\def\csname PY@tok@gp\endcsname{\let\PY@bf=\textbf\def\PY@tc##1{\textcolor[rgb]{0.00,0.00,0.50}{##1}}}
\expandafter\def\csname PY@tok@nl\endcsname{\def\PY@tc##1{\textcolor[rgb]{0.63,0.63,0.00}{##1}}}
\expandafter\def\csname PY@tok@nf\endcsname{\def\PY@tc##1{\textcolor[rgb]{0.00,0.00,1.00}{##1}}}
\expandafter\def\csname PY@tok@il\endcsname{\def\PY@tc##1{\textcolor[rgb]{0.40,0.40,0.40}{##1}}}
\expandafter\def\csname PY@tok@vi\endcsname{\def\PY@tc##1{\textcolor[rgb]{0.10,0.09,0.49}{##1}}}
\expandafter\def\csname PY@tok@no\endcsname{\def\PY@tc##1{\textcolor[rgb]{0.53,0.00,0.00}{##1}}}
\expandafter\def\csname PY@tok@w\endcsname{\def\PY@tc##1{\textcolor[rgb]{0.73,0.73,0.73}{##1}}}
\expandafter\def\csname PY@tok@gh\endcsname{\let\PY@bf=\textbf\def\PY@tc##1{\textcolor[rgb]{0.00,0.00,0.50}{##1}}}
\expandafter\def\csname PY@tok@nb\endcsname{\def\PY@tc##1{\textcolor[rgb]{0.00,0.50,0.00}{##1}}}
\expandafter\def\csname PY@tok@mh\endcsname{\def\PY@tc##1{\textcolor[rgb]{0.40,0.40,0.40}{##1}}}
\expandafter\def\csname PY@tok@ow\endcsname{\let\PY@bf=\textbf\def\PY@tc##1{\textcolor[rgb]{0.67,0.13,1.00}{##1}}}
\expandafter\def\csname PY@tok@nc\endcsname{\let\PY@bf=\textbf\def\PY@tc##1{\textcolor[rgb]{0.00,0.00,1.00}{##1}}}
\expandafter\def\csname PY@tok@s2\endcsname{\def\PY@tc##1{\textcolor[rgb]{0.73,0.13,0.13}{##1}}}
\expandafter\def\csname PY@tok@gs\endcsname{\let\PY@bf=\textbf}
\expandafter\def\csname PY@tok@sd\endcsname{\let\PY@it=\textit\def\PY@tc##1{\textcolor[rgb]{0.73,0.13,0.13}{##1}}}
\expandafter\def\csname PY@tok@kt\endcsname{\def\PY@tc##1{\textcolor[rgb]{0.69,0.00,0.25}{##1}}}

\def\PYZbs{\char`\\}
\def\PYZus{\char`\_}
\def\PYZob{\char`\{}
\def\PYZcb{\char`\}}
\def\PYZca{\char`\^}
\def\PYZam{\char`\&}
\def\PYZlt{\char`\<}
\def\PYZgt{\char`\>}
\def\PYZsh{\char`\#}
\def\PYZpc{\char`\%}
\def\PYZdl{\char`\$}
\def\PYZhy{\char`\-}
\def\PYZsq{\char`\'}
\def\PYZdq{\char`\"}
\def\PYZti{\char`\~}
% for compatibility with earlier versions
\def\PYZat{@}
\def\PYZlb{[}
\def\PYZrb{]}
\makeatother


    % Exact colors from NB
    \definecolor{incolor}{rgb}{0.0, 0.0, 0.5}
    \definecolor{outcolor}{rgb}{0.545, 0.0, 0.0}



    
    % Prevent overflowing lines due to hard-to-break entities
    \sloppy 
    % Setup hyperref package
    \hypersetup{
      breaklinks=true,  % so long urls are correctly broken across lines
      colorlinks=true,
      urlcolor=urlcolor,
      linkcolor=linkcolor,
      citecolor=citecolor,
      }
    % Slightly bigger margins than the latex defaults
    
    \geometry{verbose,tmargin=1in,bmargin=1in,lmargin=1in,rmargin=1in}
    
    

    \begin{document}
    
    
    \maketitle
    
    

    
    \hypertarget{out-of-plane-impurities-induce-deviations-from-the-monotonic-d-wave-superconducting-gap-of-cuprate-superconductors}{%
\section{Out-of-plane impurities induce deviations from the monotonic
d-wave superconducting gap of cuprate
superconductors}\label{out-of-plane-impurities-induce-deviations-from-the-monotonic-d-wave-superconducting-gap-of-cuprate-superconductors}}

\begin{longtable}[]{@{}l@{}}
\toprule
\begin{minipage}[b]{0.96\columnwidth}\raggedright
Second Report of Referee C -- LZ11166B/Wang\strut
\end{minipage}\tabularnewline
\midrule
\endhead
\begin{minipage}[t]{0.96\columnwidth}\raggedright
\textgreater{} 1) about the role of the correlations contained in the KE
mechanismwith respect to the deviation from perfect d-wave symmetry
induced by the off-diagonal scattering of the impurities\strut
\end{minipage}\tabularnewline
\begin{minipage}[t]{0.96\columnwidth}\raggedright
\textgreater{}2) how the authors arive at their form of the out-of-plane
impurity \textgreater{} scattering potential, which they write down as
\(V_0 \sum_{kk'} > (\cos (k_x - k_x')-\cos(k_y - k_y')) \tau_1\)
\textgreater{} \textgreater{}Although the authors did not even attempt
to respond seriously to \textgreater{}any of these two questions, I am
willing to accept their approach \textgreater{}to question 1) with the
argument that the comparison to a more \textgreater{}phenomenological
could be postponed to a future publication. However, \textgreater{}I am
not at all satisfied with their non-response to question 2). The
\textgreater{}standard form for the out-of-plane impurity potential of
the form
\textgreater{}\((\cos k_x - \cos k_y) + (\cos k_x' - \cos k_y')\)
corresponds to \textgreater{}a d-wave like modulation of the order
parameter on the four bonds \textgreater{}adjacent to the impurity site
as can be seen easily by Fourier \textgreater{}transformation as
follows. \textgreater{}If one assumes that the impurity sits at site
\(0\) and modifies the \textgreater{}order parameter on the four
adjacent bonds (i.e.~the bonds connecting \textgreater{}to sites
\(\pm x\) and \(\pm y\)), one would write the corresponding
\textgreater{}contribution to the Hamiltonian in real space as
\begin{eqnarray}
>H_{\rm imp}=\Delta_{\rm imp} \left( \frac{1}{2} (c_{0 \uparrow}
>c_{x \downarrow} -
>                    c_{0 \downarrow} c_{x \uparrow})
>     + \frac{1}{2} (c_{0 \uparrow} c_{-x \downarrow} -
>                    c_{0 \downarrow} c_{-x \uparrow})
>     - \frac{1}{2} (c_{0 \uparrow} c_{y \downarrow} -
>                    c_{0 \downarrow} c_{y \uparrow})
>     - \frac{1}{2} (c_{0 \uparrow} c_{-y \downarrow} -
>                    c_{0 \downarrow} c_{-y \uparrow})
>     + {\rm h.c.} \right)
>
>\end{eqnarray} By Fourier transformation to \(k\)-space one finds
\textgreater{}\begin{eqnarray} H_{\rm imp}=\sum_{kk'} c_{k \uparrow} c_{k'
>\downarrow} \Delta_{\rm imp} (( \cos k_x -\cos k'_x)-
>                 ( \cos k_y -\cos k'_y)) + {\rm h.c.}
>
>\end{eqnarray} The authors, on the other hand, start in \(k\)-space
\textgreater{}from an expression like
\begin{eqnarray} H_{\rm imp}&=&\sum_{kk'}
>c_{k \uparrow} c_{k' \downarrow} \Delta_{\rm imp} ( \cos (k_x -k'_x)-
>                  \cos (k_y -k'_y)) + {\rm h.c.}
>
>\nonumber \\ &=&\sum_{kk'} c_{k \uparrow} c_{k' \downarrow}
>\Delta_{\rm imp} \left( \frac{1}{2} \left(e^{i k_x} e^{-i
>k'_x}+e^{-i k_x} e^{i k'_x} \right) -\frac{1}{2} \left(e^{i k_y}
>e^{-i k'_y}+e^{-i k_y} e^{i k'_y}\right) \right) + {\rm h.c.}
>\end{eqnarray} and by Fourier transformation to real space I
\textgreater{}would obtain
\begin{eqnarray} H_{\rm imp}=\Delta_{\rm imp} \left(
>\frac{1}{2} \left(c_{x \uparrow} c_{-x \downarrow}
>                       - c_{x \downarrow} c_{-x \uparrow} \right)
>     - \frac{1}{2} \left(c_{y \uparrow} c_{-y \downarrow}
>                       - c_{y \downarrow} c_{-y \uparrow} \right)
>+ {\rm h.c.} \right) \end{eqnarray} \textgreater{} \textgreater{}This
would correspond to a pairing between next-next nearest neighbor
\textgreater{}sites \(x\) and \(-x\) or \(y\) and \(-y\), i.e., not to
the conventional \textgreater{}\(d_{x^2-y^2}\)-pairing. This would imply
that the impurity not simply \textgreater{}modulates the actual pairing
interaction present in the cuprates \textgreater{}but rather introduces
a different pairing interaction which leads \textgreater{}to a new local
order parameter of different symmetry. If this is \textgreater{}the
physical situation, the authors want to address, the authors
\textgreater{}should be more explicit about this point and provide a
physical \textgreater{}explanation, why they believe the impurity should
behave this way. \textgreater{} \textgreater{}In summary, I cannot
recommend publication before this point is \textgreater{}addressed
satisfactorily. \textgreater{}\strut
\end{minipage}\tabularnewline
\bottomrule
\end{longtable}

\hypertarget{third-report-of-referee-c-lz11166bwang}{%
\subsection{Third Report of Referee C --
LZ11166B/Wang}\label{third-report-of-referee-c-lz11166bwang}}

\begin{quote}
The authors have now revised their section about the out-of plane
impurity scattering. In particular the incorrect formula for the out-of
plane impurity potential, which caused my previous confusion, has
disappeared. Instead they describe in more detail their calculation,
which seems pretty standard and correct to me. Therefore, in my opinion,
the paper is now suitable for publication in Physical Review B.
\end{quote}

\href{https://en.wikipedia.org/wiki/Cooper_pair}{The reference}

    \hypertarget{reduced-fidelity-in-kitaev-honeycomb-model}{%
\section{Reduced fidelity in Kitaev honeycomb
model}\label{reduced-fidelity-in-kitaev-honeycomb-model}}

\begin{longtable}[]{@{}l@{}}
\toprule
\endhead
Report of the Referee -- AS10422/Wang\tabularnewline
\bottomrule
\end{longtable}

\begin{quote}
The authors study the reduced fidelity and the reduced fidelity
susceptibility in the Kitaev honeycomb model. It is found that the
two-site reduced fidelity susceptibility develops a peak at the quantum
critical point of the model, thus supporting the conjecture that a local
quantity can spot a topological quantum phase transition (TQPT).
\end{quote}

\begin{quote}
This is interesting work on a topical subject, with analysis and results
presented in a transparent and accessible way. With their contribution,
the authors significantly extend what is known in the literature about
reduced fidelity and TQPTs:
\end{quote}

\begin{quote}
First, the obtained result applies to the Kitaev honeycomb model, which
is far more complex than the models that have been considered hitherto.
In particular, there is no simple mapping of the Kitaev honeycomb model
to a model with classical order, making the present results highly
nontrivial.
\end{quote}

\begin{quote}
Secondly, the analysis by the authors shows the importance of choosing a
``correct'' local quantity for pinpointing a TQPT; whereas the
single-site reduced fidelity susceptibility vanishes at the transition,
the two-site susceptibility does not and instead develop a singularity.
\end{quote}

\begin{quote}
After the authors have corrected some minor idiosyncrasies in language
and grammar, I judge that the manuscript will be well suited for
publication in the Physical Review A, and I recommend its publication.
\end{quote}

    \hypertarget{interference-and-switching-of-josephson-current-carried-by-nonlocal-spin-entangled-electrons-in-a-squid-like-system-with-quantum-dots}{%
\section{Interference and switching of Josephson current carried by
nonlocal spin-entangled electrons in a SQUID-like system with quantum
dots}\label{interference-and-switching-of-josephson-current-carried-by-nonlocal-spin-entangled-electrons-in-a-squid-like-system-with-quantum-dots}}

\begin{longtable}[]{@{}l@{}}
\toprule
\endhead
Report of Referee A -- LW12539/Wang\tabularnewline
\bottomrule
\end{longtable}

\begin{quote}
This paper deals with splitting of entangled electrons from Cooper pair
emerging from superconductor. A possible way of detecting the
entanglement of electrons in two spatially separated branches is
proposed by looking at the magnetic flux dependence of the Josephson
current in a squid geometry.
\end{quote}

\begin{quote}
\begin{itemize}
\tightlist
\item
  There is one problem concerning, on the theoretical side, the
  magnitude of the signal
\end{itemize}
\end{quote}

\begin{quote}
The problem lies in the characteristic lengthscale to observe the
effect, which is set by the Fermi wavelength (and not by the coherence
length of the superconductor, which can be quite large). In the formula
giving the magnitude of the component of the Josephson current which is
due to entangled electrons in the two arms and which is periodic with
period \(2 \phi_0\) in the magnetic field, named \(I_2\) by the authors,
there is a factor \(sin^2(k_F \delta r)/(k_F \delta r)^2\). By contrast,
in practice, the factor \(e^{- 2 \delta r \, / (\pi \xi)}\) is totally
innocuous (if working with usual BCS superconductors), despite the
exponential dependence.
\end{quote}

\begin{quote}
Already in earlier work, when the ratio from crossed Andreev reflexions
to cotunneling was studied, this was a serious issue in ref. 20, (G.
Falci, D. Feinberg, and F. W. J. Hekking, Europhys. Lett. \{\bf 74\},
306 (1995).) Apparently, the new way of detecting Cooper pair
entanglements proposed by the authors demands that the two arms of the
squid geometry are of the order of the Fermi wavelength, or smaller, for
the above mentioned factor, \(sin^2(k_F \delta r)/(k_F \delta r)^2\) to
be not too small. The Fermi wavelength is of the order of a few
nanometers.
\end{quote}

\begin{quote}
Granted that the two contributions to the current (\(I_1\) and \(I_2\))
do not have the same dependence in the gap \(\Delta\). In Eq. (9), for
large Coulomb interaction one term in \(1/U\) will be small but the
other term will make \(I_1\) behave as \(1/\Delta\), which is not the
case for \(I_2\).
\end{quote}

\begin{quote}
The factor \(\gamma\), giving the ration \(I_2/2I_1\) is not expressed
explicitly. The authors just mention, in page 3, left column, second
paragraph, that \(I_1 \sim 1/\Delta\) and \$I\_2 \sim e\^{}\{-2
\delta r/\pi \Delta\} sin\^{}2(k\_F \delta r)/(k\_F
\delta r)\textsuperscript{2 \epsilon}\{-2\}. \$ However, it is not clear
how \(I_2/I_1\) behaves as a function of the three parameters,
\(\Delta\), \(\epsilon\) and \(sin^2(k_F \delta r)/(k_F \delta r)^2\).
An expression of just the scaling behavior of \(\gamma\) with these
parameters would be more clear. The values of \(\gamma\) used in the
plots in Fig. 2 may perhaps be not achievable in practice (except
\(\gamma=0\)).
\end{quote}

\begin{quote}
The authors should explain at least briefly why their method is easier
to implement than a measurement of cross-Andreev reflexion for example.
The same problem remains in the last part of the paper, \{\it
Switching of the novel Josephson current\}.
\end{quote}

\begin{longtable}[]{@{}l@{}}
\toprule
\endhead
Reply to the comments\tabularnewline
\bottomrule
\end{longtable}

\begin{quote}
Comment1: This paper deals with splitting of entangled electrons from
Coope pair emerging from superconductor. A possible way of detecting the
entanglement of electrons in two spatially separated branches is
proposed by looking at the magnetic flux dependence of the Josephson
current in a squid geometry. There is one problem concerning, on the
theoretical side, the magnitude of the signal.
\end{quote}

\begin{quote}
The problem lies in the characteristic lengthscale to observe the
effect, which is set by the Fermi wavelength (and not by the coherence
length of the superconductor, which can be quite large). In the formula
giving the magnitude of the component of the Josephson current which is
due to entangled electrons in the two arms and which is periodic with
period \(2 \phi_0\) in the magnetic field, named \(I_2\) by the authors,
there is a factor \(sin2(k_F \delta r)/(k_F \delta r)2\). By contrast,
in practice, the factor \(e^{- 2 \delta r \, / (\pi \xi)}\) is totally
innocuous (if working with usual BCS superconductors), despite the
exponential dependence.
\end{quote}

\begin{quote}
Already in earlier work, when the ratio from crossed Andreev reflexions
to cotunneling was studied, this was a serious issue in ref. 20, (G.
Falci, D. Feinberg, and F. W. J. Hekking, Europhys. Lett. \{\bf 74\},
306 (1995).) Apparently, the new way of detecting Cooper pair
entanglements proposed by the authors demands that the two arms of the
squid geometry are of the order of the Fermi wavelength, or smaller, for
the above mentioned factor, \(sin2(k_F \delta r)/(k_F \delta r)2\) to be
not too small. The Fermi wavelength is of the order of a few nanometers.
\end{quote}

Reply1: First, we thank the Referee for the careful reading, and
pointing out the importance of the Fermi wavelength for the present
issue. We agree with the Referee on that the small Fermi wavelength is
crucial for realizing the Cooper pair splitting, same as for the crossed
Andreev reflection. However, this hurdle has been overcome in the
state-of-the-art experiments, such as Ref. 15 (L. Hofstetter, S. Csonka,
J. Nyg\{\aa\}rd, and C. Sch\{"o\}nenberger, Nature \{\bf 461\}, 960
(2009)), Ref. 16 (L. G. Herrmann \emph{et
al.}, Phys. Rev.~Lett. \{\bf 104\}, 026801 (2010)), and Ref. 17 (J. Wei
and V. Chandrasekhar, Nature Physics \{\bf 6\}, 494 (2010)). For
example, the authors of Ref.\textasciitilde{}16 have reduced the
distance between the two terminals to be only of one order larger than
the Fermi wavelength. And the authors have reported strong signals
associated with crossed Andreev reflection. This indicates that the
ratio between the Fermi wavelength and the terminal distance accessible
experimentally is already large enough for the observation on the Cooper
pair splitting. The setup proposed in our manuscript on Josephson
current does not require a severer condition than those experiments
reported in the publications mentioned above.

\begin{quote}
comment2: The factor \(\gamma\), giving the ration \(I_2/2I_1\) is not
expressed explicitly. The authors just mention, in page 3, left column,
second paragraph, that ` \(I_1 \sim 1/\Delta\) and
\(I_2 \sim e^{-2 \delta r/\pi \Delta} sin2(k_F \delta r)/(k_F \delta r)2 \epsilon^{-2}.\)
However, it is not clear how \(I_2/I_1\) behaves as a function of the
three parameters, \(\Delta\), \(\epsilon\) and
\(sin2(k_F \delta r)/(k_F \delta r)2\). An expression of just the
scaling behavior of \(\gamma\) with these parameters would be more
clear. The values of \(\gamma\) used in the plots in Fig. 2 may perhaps
be not achievable in practice (except \(\gamma=0\)).
\end{quote}

Reply2: We thank the Referee for this suggestion and add the following
explicit expression
\(\gamma \approx \frac{\pi \Delta}{(4-\pi) \epsilon } e^{- 2{\delta {\bf r}}/{\pi \xi}}\sin^2(k_{\rm F} \delta {\bf r})/ (k_{\rm F} \delta {\bf r})^2\).
For details please refer to the Summary of changes and the revised
manuscript. We notice that the prefactor to the ratio
\({\rm sin}^2(k_{\rm F} \delta {\bf r})/(k_F \delta {\bf r})^2\) can be
of order of 10, or even larger, which may have brought signal of crossed
Andreev reflection accessible experimentally as reported in publications
mentioned in Reply1, taking
\(1/(k_{\rm F} \delta {\bf r})^2 \sim 10^{-2}\). This estimation of the
prefactor is reasonable, even the renormalization around resonance has
been taken into account.

\begin{quote}
Comment3: The authors should explain at least briefly why their method
is easier to implement than a measurement of cross-Andreev reflexion for
example. The same problem remains in the last part of the paper,
\{\it Switching of the novel Josephson current\}.
\end{quote}

Reply3: As mentioned in Reply1, the difficulty for implementing the
present idea is the same as the crossed Andreev reflection. The merit of
the present setup based on the Josephson effect is that the quantity we
propose to measure is directly related to non-local entangled electrons.
We add brief discussion in the revised manuscript. For details please
see the Summary of changes and the manuscript.

\begin{quote}
Comment4: - There is a second point, much less important than the first
one, it does not impair the validity of the calculation. It has to do
with the use of perturbation theory.
\end{quote}

\begin{quote}
In Eq. (20), giving \(I_2\), the last factor carries a
\(\frac{1}{ 2 \epsilon}\) factor, which is a consequence of the
perturbation scheme (to fourth order in the tunneling Hamiltonian
amplitude), as usual for the superconducting current. This could give
the impression to the non-expert reader that \(I_2\) might become very
high for \(\epsilon\) positive but small. In fact, if all the orders in
the tunneling amplitude \(T\) were summed, this would not occur. To get
proper answer close to resonance, (\(\epsilon\) going to zero, staying
always larger than zero), perturbation theory becomes not sufficient. We
think that a word of caution might be appropriate. If \(\epsilon\) is
taken to be slightly above \(0\), so that the level is normally empty,
as stated in the introductory part, the Josephson current calculated by
some NCA type self-consistent method in Ref. 24, (S. Ishizaka, J. Stone,
and T. Ando, Phys. Rev.~B \{\bf 52\}, 8358 (1995)), is much smaller than
a plain perturbation scheme would give. Thus, the ratio \(I_2/I_1\) is
probably not as favorable in reality as what plain perturbation theory
gives.
\end{quote}

\begin{quote}
Getting the proper results to infinite orders in \(T\) can be done by
introduction of the self-energy due to hopping, as in the works by
Martin-Rodero and Levi-Yeyati for example
\end{quote}

\begin{quote}
(references:
\end{quote}

\begin{quote}
A. Mart\{'\i\}n-Rodero, A. Levy-Yeyati, and F.J. Garc\{'i\}a-Vidal,
Phys. Rev.~B \{\bf 53\}, R8891 (1996).
\end{quote}

\begin{quote}
J. C. Cuevas, A. Mart\{'\i\}n-Rodero, and A. Levy-Yeyati, Phys. Rev.~B
\{\bf 54\}, 7366 (1996)).
\end{quote}

Reply4: We agree with the Referee on this point, namely the divergence
close to the resonance is caused by the fourth-order perturbation
treatment, and can be eliminated after proper renormalization. Since
this seemingly divergent term does not harm our discussion, as already
noticed by the Referee, we adopted the perturbation scheme because of
its simple and clear physical picture. We add a brief discussion on this
point in the revised manuscript. For details please see the Summary of
changes and the manuscript.

\begin{quote}
Comment5: A few remarks:
\end{quote}

\begin{quote}
\begin{itemize}
\tightlist
\item
  In Eq. (9), \(E_k\) is not defined, but it is understood that
  \(E_k = \sqrt{\xi_k^2 + \Delta^2}\) are the quasiparticle energies of
  the superconductor. Maybe this could be written explicitly by adding
  the words ``with \(E_k = \sqrt{\xi_k^2 + \Delta^2}\)'', just after Eq.
  (9).
\end{itemize}
\end{quote}

\begin{quote}
Here are some very minor points:
\end{quote}

\begin{quote}
\begin{itemize}
\tightlist
\item
  On page three, right column, last paragraph (before the summary).
  \(I_g\) is not defined, though the reader might guess its meaning. It
  could be made clear by altering the sentence to ``It is noticed that
  there is a gap \(I_g\) in the maximal Josephson current of the total
  system when the critical current of the pilot junction \ldots{}''
\end{itemize}
\end{quote}

\begin{quote}
There is also a misprint (change ``polit'' to ``pilot'' junction).
\end{quote}

Reply5: We thank the Referee for the suggestions, and revise our
manuscript accordingly. For details please see the Summary of changes
and the manuscript.

\begin{quote}
Comment6: In conclusion, for this article to be publishable in Phys.
Rev.~Lett., the authors should give clear evidence that the method they
propose can overcome the difficulties due to the smallness of the Fermi
wavelength.
\end{quote}

Reply6: We thank the Referee for his/her careful reading of our
manuscript. However, we cannot agree with his/her comment on the
possibility of the setup proposed in our manuscript. As mentioned in the
above replies, the issue of small Fermi wavelength, which was taken as
the hurdle for the realization of the crossed Andreev reflection and
Cooper pair splitting, has been overcome in the latest experiments. The
setup proposed in our manuscript does not require a severer condition
than the other experiments. By using the Josephson effect, the present
setup is able to provide clear-cut evidence for the phenomenon of Cooper
pair splitting and a more accurate estimate on the efficiency of Cooper
pair splitting. We thus believe our contribution deserves a publication
in Phys. Rev.~Lett..

Summary of changes:

\begin{quote}
\begin{enumerate}
\def\labelenumi{(\arabic{enumi})}
\tightlist
\item
  below eq. (9), ``which'' is changed into ``with
  \(E_k = \sqrt{\xi_k^2 + \Delta^2}\). It''
\end{enumerate}
\end{quote}

\begin{quote}
\begin{enumerate}
\def\labelenumi{(\arabic{enumi})}
\setcounter{enumi}{1}
\tightlist
\item
  On page 3, left column, the end of first paragraph, we add ``Here we
  note that the divergence of \(I_2\) on \(\epsilon\) is caused by the
  perturbation treatment, which can be eliminated by proper
  renormalization {[}24, 25{]}.''
\end{enumerate}
\end{quote}

\begin{quote}
\begin{enumerate}
\def\labelenumi{(\arabic{enumi})}
\setcounter{enumi}{2}
\tightlist
\item
  On page 3, left column, middle of the second paragraph, ``, where
  \(I_1\sim {1}/{\Delta}\) for strong Coulomb interaction \(U\) and
  \(I_2\sim e^{- 2{\delta {\bf r}}/{\pi \xi}}\sin^2(k_{\rm F} \delta {\bf r})/\epsilon (k_{\rm F} \delta {\bf r})^2\)
  , where
  \(\delta {\bf r}=\delta {\bf r}_{\rm L}=\delta {\bf r}_{\rm R}\),
  \(\xi\) is the coherence length and \(k_F\) is the Fermi wave vector
  {[}7, 10{]}.'' is changed into ``. In the regime of strong Coulomb
  interaction and single particle resonate tunneling with
  \(\epsilon\ll\Delta\ll U\), one has
  \(\gamma \approx \frac{\pi \Delta}{(4-\pi) \epsilon } e^{- 2{\delta {\bf r}}/{\pi \xi}}\sin^2(k_{\rm F} \delta {\bf r})/ (k_{\rm F} \delta {\bf r})^2\)
  , where
  \(\delta {\bf r}=\delta {\bf r}_{\rm L}=\delta {\bf r}_{\rm R}\),
  \(\xi\) is the coherence length and \(k_F\) is the Fermi wave vector.
  Theoretically, the factor
  \(\sin^2(k_{\rm F} \delta {\bf r})/ (k_{\rm F} \delta {\bf r})^2\)
  sets the common condition for observing the present quantity and the
  cross Andreev reflection {[}7, 20{]}.''
\end{enumerate}
\end{quote}

\begin{quote}
\begin{enumerate}
\def\labelenumi{(\arabic{enumi})}
\setcounter{enumi}{3}
\tightlist
\item
  On page 3, right column, the beginning of last paragraph, ``It is
  noticed that there is a gap in the maximal Josephson current of the
  total system when the critical current of the polit junction is
  swept,'' is changed into ``It is noticed that there is a gap \(I_g\)
  in the maximal Josephson current of the total system when the critical
  current of the pilot junction is swept,''
\end{enumerate}
\end{quote}

\begin{quote}
\begin{enumerate}
\def\labelenumi{(\arabic{enumi})}
\setcounter{enumi}{4}
\tightlist
\item
  In the list of references, we add A. Mart\{'\i\}n-Rodero, A.
  Levy-Yeyati, and F.J. Garc\{'i\}a-Vidal, Phys. Rev.~B \{\bf 53\},
  R8891 (1996). (Ref.\textasciitilde{}25 in the revised manuscript).
\end{enumerate}
\end{quote}

\begin{quote}
\begin{enumerate}
\def\labelenumi{(\arabic{enumi})}
\setcounter{enumi}{5}
\tightlist
\item
  In Fig. 2 the styles for curves are changed slightly for a better
  presentation, and the caption is changed accordingly.
\end{enumerate}
\end{quote}

    \hypertarget{time-reversal-symmetry-broken-superconductivity-induced-by-frustrated-inter-component-couplings}{%
\section{Time-Reversal-Symmetry-Broken Superconductivity Induced by
Frustrated Inter-Component
Couplings}\label{time-reversal-symmetry-broken-superconductivity-induced-by-frustrated-inter-component-couplings}}

\hypertarget{prl-round-1}{%
\subsection{PRL Round 1}\label{prl-round-1}}

\hypertarget{report-of-referee-a-lc13464hu}{%
\subsubsection{Report of Referee A --
LC13464/Hu}\label{report-of-referee-a-lc13464hu}}

The authors of the present manuscript study theoretically the
possibility of existence of time-reversal symmetry brok en (TRSB)
superconductivity without internal magnetic field. They derive, using
Ginzburg-Landau (GL) theory, a stability condition for the TRSB state in
the form of two independent relationships between the parameters of the
GL free energy. They also investigate the consequences for a Josephson
junction formed by a narrow constriction between two three-component
superconductors.

Although they present interesting results, I believe that the manuscript
is not appropriate for publication in Physical Review Letters. The
primary reason is that it does not contains enough `new' physics, since
most of the issues have been already discussed in literature. For
example, TRSB superconductivity in connection with iron pnictides has
been extensively discussed in the literature. Also, GL approaches have
been often used (actually, the GL functional is the same as in Ref.
{[}7{]}), and the existence of multiple length scales has been
discussed. It has been also discussed the fact that TRSB superconductors
cannot be categorized neither into type-I nor type-II.

I also think that the authors have not taken full advantage of the
existing literature. There are quite a lot of papers in the literature
that are relevant to the issues they study in the present manuscript. I
will mention just one of them (a recent one) by Stanev and Tesanovic,
``Three-band superconductivity and the order parameter that breaks
time-reversal symmetry'', PRB 81, 134522 (2010), a work that is also
inspired by iron pnictides.

\hypertarget{report-of-referee-b-lc13464hu}{%
\subsubsection{Report of Referee B --
LC13464/Hu}\label{report-of-referee-b-lc13464hu}}

In this paper the authors make a very interesting report on time
reversal broken superconductivity induced by frustrated inter component
coupling in absence of magnets. The paper is topical and interesting
however it raises many questions.

\begin{enumerate}
\def\labelenumi{\arabic{enumi}.}
\item
  The authors point out in the abstract that their paper stems from
  recently discovered Fe-pnictide superconductors. However it should be
  pointed out that in Fe-pnictide superconductors, by definition,
  ferromagnetism is present. So what is the realistic system they are
  talking about or is their proposal aimed to see such type of state in
  an yet to be discovered superconductor?
\item
  The GL free energy for multicomponent superconductor as mentioned in
  eq. 1 is based on what? Is it just in analogy to a single component
  superconductor or is there a way to derive it. There seems to be no
  references cited as to how this eq. 1 came about. In the references
  cited the GL free energy for multi component superconductor is
  mentioned quite differently (Stanev et.al PRB 81, 134522).
\item
  The Jospehson current is one of the main results, the formula for
  Josephson current in a multi component superconducting system hasn't
  been mentioned anywhere. Only the result is mentioned in Eq. 16. I
  would like a similar eqn. as mentioned in M. Tinkham ch.4, page 117
  eq. 4.14. What is the meaning of chirality in case of a
  superconductor? Perhaps a figure could better explain the suppression
  of critical current.
\end{enumerate}

The paper should be accompanied via a supplementary material wherein all
the details of the various calculations should be mentioned.

\hypertarget{remarks-intended-solely-for-the-editor}{%
\subsubsection{Remarks intended solely for the
editor:}\label{remarks-intended-solely-for-the-editor}}

We thank Editors for the communication as of May 5th on our
contribution.

Referee B captured the important contributions of the present work such
as the novel Josephson effect in the present TRSB superconductivity, and
evaluated very positively our work. S/he took this work as very
interesting and topical. Meanwhile, s/he raised valuable questions and
comments on our presentation, which help us in demonstrating better the
physics addressed in the present manuscript.

Unfortunately, we found that Referee A failed in capturing the essence
of our work. He confused the concepts we discussed in the manuscript,
and confused facts on literature. The comments are given in an
outlandish context, and thus are very misleading. S/he overlooked our
new results including the novel Josephson effect in the present TRSB
superconductivity when making comments.

We are thus led to feel that the comments from Referee B are valuable,
but the report from Referee A cannot be treated as a useful one.

We believe that our present manuscript contains new physics, is very
interesting and highly topical, as phrased by Referee B, and deserves
publication in Physical Review Letters. We have revised our manuscript
in order to accommodate the comments from Referee B. As the result, the
presentation of our main results is sharpened. We would be very pleased
if Editors could reconsider our manuscript.

Thank you in advance.

Sincerely yours, Xiao Hu on behalf of all authors

\hypertarget{reply-to-referee-a}{%
\subsubsection{Reply to Referee A}\label{reply-to-referee-a}}

\begin{quote}
C1. Although they present interesting results, I believe that the
manuscript is not appropriate for publication in Physical Review
Letters. The primary reason is that it does not contains enough `new'
physics, since most of the issues have been already discussed in
literature. For example, TRSB superconductivity in connection with iron
pnictides has been extensively discussed in the literature. Also, GL
approaches have been often used (actually, the GL functional is the same
as in Ref. {[}7{]}), and the existence of multiple length scales has
been discussed. It has been also discussed the fact that TRSB
superconductors cannot be categorized neither into type-I nor type-II.
\end{quote}

R1. We cannot agree with Referee A at all. Let us check the comments one
by one, noting that this Referee has confused many facts, with an
outlandish context.

R.1.1. First, as illustrated schematically in Fig. 1, we have revealed
for the first time a novel Josephson effect of the TRSB
superconductivity. Based on an analytic calculation for a junction of
constriction, we demonstrate that the critical Josephson current between
the two TRSB states is much suppressed from the one between a same
state. This prototype constriction junction can be used to detect the
brand new TRSB superconductivity. This new and, in our opinion,
important result is clearly documented in our manuscript, and captured
by Referee B. It is strange why Referee A overlooks this part when s/he
made comments. In order to emphasize this result we add it to the title
of our paper.

R1.2. The TRSB superconductivity was discussed back in 1998 (Ref.5), and
then in 2010 after the discovery of iron-pnicited superconductors
(Ref.6), as introduced in our manuscript with respect to their
importance. However, we notice that the previous papers treated
primarily the isotropic case, a situation very rare, if not impossible,
in reality, and did not discuss whether the TRSB superconductivity
exists in a more general situation. This issue of stability is highly
nontrivial and especially important for this novel superconductivity as
discussed in our manuscript. For the first time, we formulate explicitly
the stability conditions in terms of GL theory. This new result is very
important since it is useful for future searches on candidate TRSB
superconductors from relevant materials. In order to emphasize this
result we add it to the title of our paper.

R.1.3. The GL theory is very general since it is based on the symmetry
consideration. In this sense, our GL free-energy functional should not
be different from previous works FORMALLY. It is therefore not
surprising that Eq.1 is similar to those in previous works. The unique
contribution of the present work is to reveal novel properties and the
conditions for them to appear, in terms of explicit relations among the
coefficients of the GL functional. Note even the ground-breaking work by
Abrikosov on vortex states in type II superconductivity started from the
same free-energy functional used by Ginzburg and Landau previously.

R.1.4. The existence of multi divergent lengths of superconductivity
coherence has not been addressed in literature, oppositely to the
comment from Referee A (Ref.7 even did not touch upon this issue). It
should be emphasized that, there are no multi divergent lengths of
superconductivity coherence in two-component superconductivity, needless
to say their absence in single-component one. It is also impossible in a
three-component TRS-reserved case. The existence of multi divergent
lengths of superconductivity coherence, besides the London penetration
depth, is a property unique to the TRSB superconductivity, as revealed
for the first time in the present work, oppositely to the comment of
this Referee. Referee A confused the multi divergent lengths of
superconductivity coherence with multi length scales, the latter was
recently discussed for two-component case. Here we wish to make it clear
that, since GL theory is in principle justified close to Tc with small
order parameter(s), the important length scales are the divergent ones.
In two-component superconductors, there is only one divergent
coherence-length for a finite inter-component coupling, which means
directly that close to Tc a two-component superconductor behaves as a
single-component one. As addressed in the present manuscript, the
present TRSB superconductivity realized in systems of three components
is really a novel state with unique properties beyond conventional
theory developed for single-component superconductivity.

R.1.5. We have calculated explicitly the nucleation magnetic field, the
thermodynamic critical field, and the coherence length. With these
analytic results, we conclude that the TRSB superconductivity cannot be
categorized into type I and II according to the GL number kappa. The
results are not available in any literature, and the failure of
categorization of TRSB superconductivity into type I and II has not yet
been discussed in literature so far, oppositely to the comment from this
Referee.

R.1.6. We notice that the present TRSB superconductivity exhibits two
divergent lengths associated with the coherence of superconductivity,
besides the London penetration depth, in contrast to two-component
superconductors. Therefore, the possibility that vortices attract each
other when far away (but repel when close) occurs in all the temperature
regime below critical point, in a sharp contrast to two-component cases.

With all these novel results, we believe that the present contribution
contains new and important physics, as captured by Referee B and clearly
phrased in the report.

\begin{quote}
C2: I also think that the authors have not taken full advantage of the
existing literature. There are quite a lot of papers in the literature
that are relevant to the issues they study in the present manuscript. I
will mention just one of them (a recent one) by Stanev and Tesanovic,
``Three-band superconductivity and the order parameter that breaks
time-reversal symmetry'', PRB 81, 134522 (2010), a work that is also
inspired by iron pnictides.
\end{quote}

R2. We have cited the papers mostly relevant to the present work, and
noticed clearly their importance to the field in the very beginning of
our manuscript. The work by Stanev and Tesanovic, ``Three-band
superconductivity and the order parameter that breaks time-reversal
symmetry'', PRB 81, 134522 (2010), was also cited (Ref. 6). We wish to
make it clear that the results presented in this manuscript are
original, and in our opinion, go beyond the previous works. If this
Referee thinks the reference list is not complete, we would be very
happy to learn from this Referee on the paper(s) we should include. If
this Referee claims that a part of the present work was already
addressed by the paper of Stanev and Tesanovic, or other papers, s/he is
better to indicate it clearly.

In short, we cannot agree with the comments from this Referee which are
biased by misunderstanding. We would be pleased if this Referee
considers the results presented in this manuscript more carefully.

\hypertarget{reply-to-referee-b}{%
\subsubsection{Reply to Referee B}\label{reply-to-referee-b}}

\begin{quote}
C0: In this paper the authors make a very interesting report on time
reversal broken superconductivity induced by frustrated inter component
coupling in absence of magnets. The paper is topical and interesting
however it raises many questions.
\end{quote}

R0: First of all, we thank Referee B for the very nice words on our
contribution. For the questions we try to answer one by one in what
follows.

\begin{quote}
C1. The authors point out in the abstract that their paper stems from
recently discovered Fe-pnictide superconductors. However it should be
pointed out that in Fe-pnictide superconductors, by definition,
ferromagnetism is present. So what is the realistic system they are
talking about or is their proposal aimed to see such type of state in an
yet to be discovered superconductor?
\end{quote}

R1. In the recently discovered iron-pnictide superconductors, there are
strong antiferromagnetic fluctuations contributed from d-orbits of Fe
atoms, and there is no ferromagnetic order in these systems. Actually,
it is the very breakthrough brought about to the field of
superconductivity by the discovery of iron-pnictide superconductors;
before the discovery, it was generally believed that Fe atoms would
contribute to a ferromagnetic order which tends to break Cooper pairs,
and thus suppresses superconductivity.

Meanwhile, we would like to notice that there is no established
microscopic theory, and no confirmed experimental evidence either, for
the coexistence of magnetism and superconductivity in these materials.
Therefore, it is natural not to include a magnetic part in the GL
free-energy functional. The time-reversal symmetry breaking discussed
here is purely caused by the frustrated couplings among superconducting
components. It is unique since it is not accompanied by an internal
magnetic field like a triplet superconductor.

\begin{quote}
C2. The GL free energy for multicomponent superconductor as mentioned in
eq. 1 is based on what? Is it just in analogy to a single component
superconductor or is there a way to derive it. There seems to be no
references cited as to how this eq. 1 came about. In the references
cited the GL free energy for multi component superconductor is mentioned
quite differently (Stanev et.al PRB 81, 134522).
\end{quote}

R2. References were cited just below Eq.1 (Ref.s 5, 7, 12 and 13). In
Ref. 13, the detailed GL free-energy functional was given explicitly for
a two-component superconductor by the quantities of BCS theory. A very
similar expression was explicitly given in Ref. 5. The extension to
three components is straightforward. In our manuscript, we use the
simplest version including only inter-band Josephson-like couplings,
since first they are the lowest-order, and thus the most prominent,
inter-band interactions, and second the novel features can be captured
by the minimal free-energy functional (Eq.1 in our manuscript), and
third we have made sure that including more terms will not change the
novel physics addressed here. We mentioned this at the end of paragraph
of Eq. 1.

The GL free-energy functional can also be composed in analogy to single
component superconductor by including possible inter-component
couplings, as already noticed by Referee, benefitting from the great
concept of GL theory on the order parameters. In this sense, the
expression of GL free-energy functional itself is very much general. The
unique contribution of the present work is to reveal novel properties
and the conditions for them to appear, in terms of relations among the
coefficients of the GL functional.

Please be noticed that the paper by Stanev et al. (Ref. 6) did not
discuss the GL free-energy functional. They discussed the free energy
within a simplified BSC scheme.

\begin{quote}
C3. The Jospehson current is one of the main results, the formula for
Josephson current in a multi component superconducting system hasn't
been mentioned anywhere. Only the result is mentioned in Eq. 16. I would
like a similar eqn. as mentioned in M. Tinkham ch.4, page 117 eq. 4.14.
What is the meaning of chirality in case of a superconductor? Perhaps a
figure could better explain the suppression of critical current. The
paper should be accompanied via a supplementary material wherein all the
details of the various calculations should be mentioned.
\end{quote}

R3. First, we thank Referee B very much for the comment on our
presentations. In order to emphasize the importance of the Josephson
effect in the present TRSB superconductivity, we add it into the title
of our manuscript. We have also included more descriptions on the
derivations of our results. First, we add a short note for one of the
stability conditions below Eq.4. Second, we add a note in Ref. 16 on the
absence of degenerated root for attractive (positive in our definition)
inter-component couplings, which helps one to understand the emergence
of the TRSB solution. Third, we add a short note below Eq.12 for the two
divergent lengths of superconductivity coherence, pointing clearly that
it is related with the doubly degenerated root at Tc as in Fig.2.
Fourthly, we have revised the description on derivation of the wave
functions in the constriction junction (please see the paragraph
starting with Finally let us investigate the Josephson current .
Fifthly, we have included the formula on the Josephson current for
three-component superconductors (see Eq.16), following the suggestions
of Referee B. These revisions make the derivations of our main results
accessible reasonably.

Meanwhile we believe that other minor details are almost
straightforward, and do not warrant a supplementary material. We would
be very much pleased if Referee B could reconsider this suggestion.

We appreciate Referee B very much for the comment on chirality, which we
borrowed from one of the references (Ref. 7). Indeed, chirality in
physics usually refers to the situation when a phenomenon is not
identical to its mirror image, and thus is irrelevant to the present
system of superconductivity. We therefore refrain from using the concept
of chirality in our manuscript. Please let us note that the physics
addressed here does not change with this revision.

As for the suppression of Josephson current between two TRSB states, we
notice that the total Josephson current is just the summation of
Josephson currents for individual components, as in Eq.16. For the TRSB
cases illustrated in Fig. 1, it is easy to see that a large cancellation
occurs among those for individual components, and for the isotropic
case, namely all components share the same strength and the same
coupling with other components, the total Josephson current is
suppressed to zero. Please let us notice that the physics discussed here
remains the same even including full inter-component couplings. We
believe that the physics here has been demonstrated quite clearly (Eqs.
16 and 17, and Fig.1), and an additional figure seems duplicated.

Finally, we would like to thank Referee B for his critical reading,
valuable comments, and very positive assessment on our contribution. The
criticisms certainly help us to sharpen our presentation. We sincerely
hope that our revisions can accommodate the comments, and appreciate
very much the recommendation from this Referee.

\hypertarget{summary-of-changes}{%
\subsubsection{Summary of changes:}\label{summary-of-changes}}

1.We modify slightly the title of our paper by adding Stability and
Novel Josephson Effect of, in order to sharpen the presentation of our
work. 2.We delete the description of chirality in abstract,
introduction, caption of Fig. 1 and in the part of discussion on
Josephson current below Eq.17. 3.We add a short note for one of the
stability conditions below Eq.4. 4.We add a note in Ref. 16 on the
absence of degenerated root for attractive (positive in our definition)
inter-component couplings, which helps one to understand the emergence
of the TRSB solution. 5.We add a short note below Eq.12 for the two
divergent lengths of superconductivity coherence. 6.At the end of the
paragraph for London penetration depth, thermodynamics critical field,
and nucleation field (Eqs.13-15), we add some discussions on the
comparison between the present TRSB superconductivity and two-component
cases. 7.Minor changes are made in the paragraph starting from Finally
let us investigate the Josephson current, which make the derivations
more accessible. 8.We include in Eq.16 the formula for Josephson current
derived from the GL free-energy functional. 9. Several typos are
corrected. 10. PACS numbers are slightly changed.

\hypertarget{prl-round-2}{%
\subsection{PRL Round 2}\label{prl-round-2}}

\hypertarget{second-report-of-referee-b-lc13464hu}{%
\subsubsection{Second Report of Referee B --
LC13464/Hu}\label{second-report-of-referee-b-lc13464hu}}

The authors in this revised manuscript have made changes to make the
paper more accessible and better. However, a lot still needs to be done.
More importantly there are some mistakes I discovered along with a few
unanswered questions which will have a bearing on whether the work
deserves to be a PRL. I maintain that it does have seeds of thought
which are genuinely exciting but the present version does not fulfill
many of the conditions for being a PRL. I hope the authors ponder of the
following issues raised and diligently answer them.

\begin{enumerate}
\def\labelenumi{\arabic{enumi}.}
\item
  There is an error in Eq. 17. A calculation by me shows Ic=sqrt(\sum\_j
  i\_j\^{}2+2i\_1 i\_2 cos 2phi\_21 +2 i\_1 i\_3 cos 2 phi\_31 +2 i\_3
  i\_2 cos(2 phi\_21 - 2 phi\_31)).
\item
  It is incorrect to state I\_c =0 in isotropic case. My calculation
  reveals, isotropic case defined as (I think!) phi\_21=phi\_31 implies
  I\_c not equal to zero.
\item
  Nowhere is the isotropic case properly defined only a line in the 2 nd
  paragraph of page 1 attempts to explain it. A detailed contrast,
  parameter wise, with anisotropic case should be presented.
\item
  I had specifically asked the authors to explain Eq. 1 properly. In the
  revised manuscript they do not change anything as regards eq 1. They
  should make it as self contained as possible and not by glibly
  referring to 5,7,12, 13. This is a PRL not a working note in progress.
\item
  A perspective is missing in the conclusion. It is quite arbitrary. It
  states that although GL approach is justified only close to critical
  pt. somehow the phenomena are expected to hold for the whole
  temperature regime. why?
\item
  Finally the authors should compare the results of their work with
  other works on Iron-pnictide and/or ferromagnetic superconductors to
  reveal pairing symmetries. Examples are many, few of note are:
\end{enumerate}

A. Fe-pnictide - Linder, sudbo, PRB 79, 020501(2010); Araujo,
Sacramento, PRB 79, 174529(2009)

B. Ferromagnetic Superonductors - Benjamin, PRB 74, 180502(2006);
Bolech, Giamarchi, PRL 92, 127001 (2004).

\hypertarget{to-editor}{%
\subsubsection{To Editor:}\label{to-editor}}

Thank you for the message as of June 4 on our manuscript LC13464
``Stability and Novel Josephson Effect of Time-Reversal-Symmetry-Broken
Superconductivity Induced by Frustrated Inter-Component Couplings''.
Referee B maintains his/her evaluation on our contribution as having
``seeds of thoughts'' and thus ``genuinely exciting''. The main point of
this second report is a question on our result on critical Josephson
current (Eq.17). It turns out that his/her calculation reproduces our
result exactly, noticing that s/he missed the sign carried by the phase
difference. Referee B is highly knowledgeable in this field, and his
assessment is really important. On the other hand, we wish to bring a
technical point to the attention of Editor, namely most of the questions
or comments in the second report could well be raised, and thus
addressed, in the first round. We would therefore beg the Editor taking
this factor into account when treating our present manuscript. Thank you
very much for your kind consideration in advance.

Sincerely yours, Xiao Hu

\hypertarget{to-referee-b}{%
\subsubsection{To Referee B:}\label{to-referee-b}}

\begin{quote}
C0. The authors in this revised manuscript have made changes to make the
paper more accessible and better. However, a lot still needs to be done.
More importantly there are some mistakes I discovered along with a few
unanswered questions which will have a bearing on whether the work
deserves to be a PRL. I maintain that it does have seeds of thought
which are genuinely exciting but the present version does not fulfill
many of the conditions for being a PRL. I hope the authors ponder of the
following issues raised and diligently answer them.
\end{quote}

R0. We are sincerely grateful for Referee B, who notices that our
revisions improve the manuscript, and maintains the high evaluation on
our work. We are sorry for that part of our previous answers could not
satisfy Referee B, and try to accommodate his/her comments in the
present reply and revision. As for the new comment on a possible
mistake, please let us clarify that it is not the case, as discussed in
what follows.

\begin{quote}
C1. There is an error in Eq. 17. A calculation by me shows
Ic=sqrt(\sum\_j i\_j\^{}2+2i\_1 i\_2 cos 2phi\_21 +2 i\_1 i\_3 cos 2
phi\_31 +2i\_3 i\_2 cos(2 phi\_21 - 2 phi\_31)).
\end{quote}

R1. The calculation by Referee B is exactly the same as Eq.17 noticing
phi\_21-phi\_31=phi\_23. Therefore, there is no error in Eq.17.

\begin{quote}
C2. It is incorrect to state I\_c =0 in isotropic case. My calculation
reveals, isotropic case defined as (I think!) phi\_21=phi\_31 implies
I\_c not equal to zero.
\end{quote}

R2. For the isotropic case, phi\_21= - phi\_31=2 pi/3, where the phase
difference carries a sign. Plugging this into Eq.17 results in I\_c=0.

\begin{quote}
C3. Nowhere is the isotropic case properly defined only a line in the 2
nd paragraph of page 1 attempts to explain it. A detailed contrast,
parameter wise, with anisotropic case should be presented.
\end{quote}

R3. The isotropic case was mentioned in the paragraph starting with
``Next we investigate the coherence length \ldots{}'' (the paragraph
before the one including Eq.11). We apologize for that the definition on
the isotropic/anisotropic cases was not clear enough in our previous
versions. In the present study, an isotropic system is defined with the
coefficients in Eq.1 as a\_1=a\_2=a\_3=a, b\_1=b\_2=b\_3=b, and
gamma\_12 =gamma\_13 =gamma\_23 =gamma. The phase differences are given
by \phi\_21= phi\_32= phi\_13= 2 pi/3 (with phi\_jk=-phi\_kj). In an
anisotropic system, some of the coefficients are different in different
bands in Eq.1. For detailed revisions, please refer to the manuscript,
the 2 nd paragraph of the manuscript and the paragraph starting with
``Next we investigate the coherence length \ldots{}'' .

\begin{quote}
C4. I had specifically asked the authors to explain Eq. 1 properly. In
the revised manuscript they do not change anything as regards eq 1. They
should make it as self contained as possible and not by glibly referring
to 5,7,12, 13. This is a PRL not a working note in progress.
\end{quote}

R4. We are sorry for that we only noticed the citations in our previous
reply. As Referee B put, Eq.1 is the very starting point of the present
work, and is thus very important. We revise our description on Eq.1,
following the comment of Referee B. For details please see the revised
manuscript around Eq.1.

\begin{quote}
C5. A perspective is missing in the conclusion. It is quite arbitrary.
It states that although GL approach is justified only close to critical
pt. somehow the phenomena are expected to hold for the whole temperature
regime. why?
\end{quote}

R5. We include a paragraph a the last part of the manuscript for
perspective, as suggested by Referee B, noticing two possible intriguing
future directions: (1) vortex state of this TRSB superconductivity
should be interesting due to the novel phase distribution among
components, and (2) tunneling with other TRSB materials, such as triplet
superconductor and/or magnetic materials. We delete the discussion on
the GL theory at lower temperature. Please see the last paragraph of the
manuscript.

As for the question from Referee B, let us mention that many phenomena
predicted by GL theory, which is only justified for temperature close to
Tc, have been observed down to low temperature regime. A good example is
the Abrikosov vortex lattice state, which is derived primarily based on
GL theory, but turns out to occupy the whole temperature regime. The
reason for this is primarily due to the fact that the system is governed
by the ratio between coherence length and penetration depth, which is
temperature independent. Another reason could be that the corrections to
the GL theory only cause small deviations. But this discussion is beyond
the scope of the present paper.

\begin{quote}
C6. Finally the authors should compare the results of their work with
other works on Iron-pnictide and/or ferromagnetic superconductors to
reveal pairing symmetries. Examples are many, few of note are:
\end{quote}

\begin{quote}
A. Fe-pnictide - Linder, sudbo, PRB 79, 020501(2010); Araujo,
Sacramento, PRB 79, 174529(2009)
\end{quote}

\begin{quote}
B. Ferromagnetic Superonductors - Benjamin, PRB 74, 180502(2006);
Bolech, Giamarchi, PRL 92, 127001 (2004).
\end{quote}

R6. As pointed out by Referee B, tunneling phenomena have been
investigated extensively so far, for triplet superconductor by Bolech
and Giamarchi, those involving interface between single-band metal and
iron-based superconductor by Linder and Sudbo, and Araujo and
Sacramento, and that of a ferromagnetic insulator by Kawabata et al.~In
contrast to the previous works, our setup of Josephson junction is based
on a constriction structure, with a short and thin bridge between two
bulks of otherwise a single superconducting material. In this system,
Cooper pairs transmit through the bridge from three bands to three bands
on the two sides. The wave functions are described by the GL equations,
and the Josephson current can be evaluated directly, as in Eq.17,
following the idea by Larkin for single-component superconductivity
nicely documented in the book by Tinkham. As revealed in our manuscript,
the result is however very interesting due to the interference among
multi components. The present novel Josephson effect is caused by the
TRSB property of the composite wave function, not by the pairing
symmetry of individual components. We add a new paragraph (the one
before the last paragraph of the manuscript) for this discussion.

In short, we really appreciate the comments from Referee B, which help
us improve much our presentation. We have revised our manuscript
following the comments by Referee B, namely giving a detailed
description on GL functional, a comparison between the present Josephson
effect and previous works, and a perspective. On the other hand, we
notice that the calculation by Referee B reproduces exactly our result
on critical Josephson current, noting that the phase difference carries
a sign. We sincerely hope that our revisions accommodate all the
criticisms of Referee B, and would be very pleased if s/he could
recommend the publication of our contribution in PRL.

\hypertarget{summary-of-changes-1}{%
\subsubsection{Summary of changes:}\label{summary-of-changes-1}}

1)In the 1-st paragraph 1-st sentence `without internal magnetic field'
is included. 2)In the 4-th paragraph, the description on `anisotropy
system' is deleted, instead `we consider generally a system of
un-equivalent components \ldots{}' is included. 3)The paragraph
including Eq.(1) has been revised, including more description on the
free-energy functional. 4)In the paragraph above Eq.(11), a definition
of the isotropic system is given. The phase differences
\phi\_21=-\phi\_31=2\pi/3 are given explicitly. 5)A new paragraph is
added before the concluding paragraph for the comparison between the
present Josephson effect and previous works on tunneling phenomena. 6)A
perspective is given in the last concluding paragraph. 7)Some references
are added (Refs. 19-22). 8)Abstract and acknowledgement are shortened
and some typos are corrected.

\hypertarget{prl-round-3}{%
\subsection{PRL Round 3}\label{prl-round-3}}

\hypertarget{third-report-of-referee-b-lc13464hu}{%
\subsubsection{Third Report of Referee B --
LC13464/Hu}\label{third-report-of-referee-b-lc13464hu}}

The authors have made changes but it is not enough.

Some more corrections are necessary.

\begin{enumerate}
\def\labelenumi{\arabic{enumi}.}
\item
  The fact phi\_21 - phi\_31 = phi\_23 mentioned in the response isn't
  mentioned anywhere in the manuscript.
\item
  With the isotropic condition mentioned I\_c = sqrt(i\_1 (i\_1 - i\_2)
  + i\_2 (i\_2 - i\_3)+ i\_3 (i\_3 -i\_1)). This will only be zero when
  i\_1 = i\_2 =i\_3. This condition can be met even in anisotropic case
  depending on parameters. Further the authors in the response reveal
  that result reveals interference among components. As mentioned i\_1
  =i\_2 =i\_3 implies one and one thing only absence of interference
  among components. How do they explain the contradiction.
\item
  In 4th para, un-equivalent should be replaced by inequivalent.
\item
  Although a perspective on tunneling is welcome, the authors have
  completely ignored other methods to detect pairing symmetries in
  ferromagnetic or pnictide superconductors. Especially comparison with
  crossed andreev reflection and/or josephson effect with
  non-superconducting material sandwiched between superconductors is
  lacking. I wonder why? Since in their system there is no
  non-superconducting materials the benefits as well as deficiencies of
  their method as compared with others should be brought forward.
\end{enumerate}

All in all the authors changes are cosmetic they lack a real willingness
to modify and correct the manuscript.

\hypertarget{referee-c-lc13464hu}{%
\subsubsection{Referee C -- LC13464/Hu}\label{referee-c-lc13464hu}}

I agree with all the comments of both previous referees, and reach the
same conclusion: the paper is not suitable for publication in Phys.
Rev.~Lett. The present version might be published as a Brief Report, but
in my opinion the authors should better expand it into a full-size Phys.
Rev.~B article, addressing in the process the criticisms raised by the
referees.

The paper is an extension of the Phys. Rev.~B by Agternerg et al. (Ref.
5 of the manuscript) and includes several interesting results that, once
properly explained and based, deserve publication; Ref. 5 considered the
isotropic case, the present authors address a more general
Landau-Ginzburg (LG) functional. However, whereas the paper by Agterberg
et al.~is built on a physical system, elaborating on the underlying
symmetries that lead to the LG expression they treat, the present paper
simply assumes a certain LG expression and then minimizes it. From the
perspective of the reader, the authors could have considered a
magnetically-ordered system (with several competing interactions) as
well. I give as an example the works of D. Mukamel in the 70's on
(magnetic) coupled x-y models.

Indeed, both referees, in particular B, point out that the LG expression
is introduced out of thin air, without relating it to any specific
system (even after the second submission!). In that context, I find the
reply of the authors to Referee A unacceptable :``However, we notice
that the previous papers treated primarily the isotropic case, a
situation very rare, if not impossible, in reality, and did not discuss
whether the TRSB superconductivity exists in a more general situation.''
The present article does not offer any relation to a real physical
system, except for announcing right at the beginning that it has to do
with ``our understanding of the birth of the universe (!)''.
Furthermore, referee A commented that the fact that such superconductors
cannot be categorized neither into type I nor into type II has been
already discussed in the literature. Instead of properly answer him, the
authors chose to declare ``the failure of categorization of TRSB
superconductivity into type I and II has not yet been discussed in
literature so far, oppositely to the comment from this Referee.'' This
is simply not true. Typing ``Type 1.5 superconductivity'' in Google
yielded 428,000 entries. It might be that the reason is different than
in the present analysis, but still one expects the authors to comment on
the difference.

Whereas the authors address the genuine comments of Referee A rather
aggressively (for no ap- parent reason), they treat Referee B extremely
politely, but nonetheless, fail to supply the required clarifications.
For example, referee B pointed out the possibility of a magnetic order
in the pnictides, which is not included in the GL expression (1) of the
manuscript. As the only physical system they mention is the iron
pnictides, I find the authors' reply ``Meanwhile, we would like to
notice that there is no established microscopic theory, and no confirmed
experimental evidence either, for the coexistence of magnetism and
superconductivity in these materials. Therefore, it is natural not to
include a magnetic part in the GL free-energy functional. The
time-reversal symmetry breaking discussed here is purely caused by the
frustrated couplings among superconducting components. It is unique
since it is not accompanied by an internal magnetic field like a triplet
superconductor.'' ) very unsatisfactory. What is then the reason for the
frustrated couplings? in which physical system can they be realized? (I
am echoing the comments of referee B.)

I also find their comment (both in the manuscript and in their reply to
referee B) ``and third we have made sure that including more terms---in
the GL expression-- will not change the novel physics addressed here.''
What do they mean? which additional terms did they check? Equation (1)
of the manuscript includes just one possible combination of the
fourth-order terms, out of many more (see Ref. 5 in this context), for
example, terms that couple the three order parameters.

In summary, the paper presents an interesting analysis of the GL
expression which includes three superconducting order-parameters with
competing interactions. The authors are advised to revise their work
along the lines proposed by the previous referees. I think that they
should relate their work to actual systems, and also to similar problems
appearing in the context of magnetism (i.e., coupled x-y models, and
maybe even multi-ferroics).

\hypertarget{third-report-of-referee-b}{%
\subsubsection{Third report of Referee
B:}\label{third-report-of-referee-b}}

\begin{quote}
C1. The fact phi\_21 - phi\_31 = phi\_23 mentioned in the response isn't
mentioned anywhere in the manuscript.
\end{quote}

A1: We took this as trivial. As it may cause confusion, we include this
relation into manuscript.

\begin{quote}
C2. With the isotropic condition mentioned I\_c = sqrt(i\_1 (i\_1 -
i\_2) + i\_2 (i\_2 - i\_3)+ i\_3 (i\_3 -i\_1)). This will only be zero
when i\_1 =i\_2 =i\_3. This condition can be met even in anisotropic
case depending on parameters. Further the authors in the response reveal
that result reveals interference among components. As mentioned i\_1
=i\_2 =i\_3 implies one and one thing only absence of interference among
components. How do they explain the contradiction.
\end{quote}

A2: For isotropic case, the total critical current I\_c is suppressed to
zero due to cancellations among the tunneling currents of individual
components. We used the term ``interference'' for this cancellation in
our last response (R6). We apologize if the term ``interference''
brought confution, but please let us make it clear that there is no
contradiction here.

As noticed by Referee, i\_1=i\_2=i\_3 can be achieved in anisotropic
cases since they are determined by phi\_j and m\_j. But if psi\_j are
different for the three components, phase differences phi\_jk should be
different from 2 pi/3, and thus one cannot have the relation I\_c =
sqrt(i\_1 (i\_1 - i\_2) + i\_2 (i\_2 - i\_3)+ i\_3 (i\_3 -i\_1)) any
more.

\begin{quote}
C3. In 4th para, un-equivalent should be replaced by inequivalent.
\end{quote}

A3: We thank the Referee for the kind proof reading. We correct it in
the revised manuscript.

\begin{quote}
C4. Although a perspective on tunneling is welcome, the authors have
completely ignored other methods to detect pairing symmetries in
ferromagnetic or pnictide superconductors. Especially comparison with
crossed andreev reflection and/or josephson effect with
non-superconducting material sandwiched between superconductors is
lacking. I wonder why? Since in their system there is no
non-superconducting materials the benefits as well as deficiencies of
their method as compared with others should be brought forward. All in
all the authors changes are cosmetic they lack a real willingness to
modify and correct the manuscript.
\end{quote}

A4: We think the constriction junction structure is the best setup for
detecting the TRSB superconducting state, wish to make it clear that
this constriction junction is not designed for detecting the pair
symmetry of iron-pnictide superconductivity. In what follows we discuss
briefly whether Andreev reflection and other junctions are suitable for
detecting the TRSB state.

Andreev reflection is sensitive to spin state of quasiparticle, and thus
symmetry of Cooper pairing. Therefore, it is very useful for
investigating the p-wave superconductor such as Sr\_2RuO\_4. We don't
think a conventional experimental set up for Andreev reflection can
detect the phases of superconductivity gap functions, which is the most
important quantity in the present TRSB state. The phases of gap
functions can best be measured by Josephson junctions.

S/I/S junctions may detect the present TRSB states in principle.
However, the S/I interfaces cause scatterings between different
components, and as a result, the difference in critical Josephson
currents for the two possible configurations (same TRSB state vs
opposite TRSB states) diminishes, making experimental sensitivity worse,
as compared with the constriction junction with short bridge proposed in
our work.

S/M/S junctions and S/F/S junctions may provide rich physics for the
present TRSB state. But the tunneling properties between multi-component
superconductor and single-band metal are a matter of recent
investigations. Without a well-established understanding on them, a
S/M/S hardly provides a clear detection of the present TRSB
superconductivity. It is a similar, or even a worse situation for S/F/S
junction, since the ferromagnetic layer causes scatterings which break
the spin-singlet pairing for individual components for the present TRSB
state. We include these discussions briefly into the concluding part of
the manuscript.

We apologize for that the last revisions are not good enough. It is
because we had no insightful understanding, rather than due to lacking
willingness to improve the manuscript. We hope these explanations and
revisions to the manuscript can satisfy this Referee, and appreciate
very much his/her recommendation.

\hypertarget{third-report-of-referee-c}{%
\subsubsection{Third report of Referee
C:}\label{third-report-of-referee-c}}

\begin{quote}
C1: The paper is an extension of the Phys. Rev.~B by Agternerg et al.
(Ref. 5 of the manuscript) and includes several interesting results
that, once properly explained and based, deserve publication; Ref. 5
considered the isotropic case, the present authors address a more
general Landau-Ginzburg (LG) functional. However, whereas the paper by
Agterberg et al.~is built on a physical system, elaborating on the
underlying symmetries that lead to the LG expression they treat, the
present paper simply assumes a certain LG expression and then minimizes
it. From the perspective of the reader, the authors could have
considered a magnetically-ordered system (with several competing
interactions) as well. I give as an example the works of D. Mukamel in
the 70's on (magnetic) coupled x-y models.
\end{quote}

A1: We take the work by Agternerg et al.~as an important literature,
which proposed for the first time the possibility of the TRSB
superconductivity caused by frustrated inter-component coupling for
systems of multi components, and cite this paper in the introductory
part. As correctly noted by this Referee, Agterberg et al discussed the
underlying symmetry and arrived at the GL expression. It would be best
if one could do the similar analysis for the recent booming
iron-pnicitide superconductivity. Unfortunately, it is impossible at
this moment since a microscopic theory for this new high-temperature
superconductivity is not available, and thus there is no counterpart of
the BCS theory used by Agterberg. We thus ask ourselves a different
question, namely what are the general conditions at the GL level for a
stable TRSB superconductivity with nontrivial phase differences among
components. We found that this question can be answered even without
knowing the detailed microscopic mechanism, and the TRSB state may be
detected by experiments. While we expect the iron-pnictide
superconductors, a class of superconductors having more than three
components and thus motivating our study, provide a good chance for
testing this interesting state, our theory covers more general cases. By
the way, the GL expression (1) is not just an assumption. It can be
derived from BCS theory for the case that the individual components are
of BCS-like s-wave pairing symmetry. The procedures are available in
literatures (the review by Gurevich for two-component superconductivity
is cited as Ref.13).

As for the relation with magnetism, we have different understanding with
this Referee. It is well known that the concept of frustration has been
widely used for spin systems. For example, in the xy model on triangular
lattice with frustrated exchange coupling, spin vectors on the three
sublattices take the so-called 120-degree configuration. But this
configuration is in the real space, and the phase transition between
this state and the high-temperature paramagnetic phase is associated
with the breaking of the O(n) symmetry and translational symmetry. If
the spin vectors are equivalent at all sites, no magnetization appears
in the low-temperature phase, and thus in any sense the time-reversal
symmetry is unbroken. In a sharp contrast, in superconductivity, the
three components reside simultaneously at every point in the real space.
The 120-degree phase differences take place in the k-space, rather than
in the real space as in magnetic systems. The phase transition is
associated with the breakings of the U(1) symmetry and
time-reversal-symmetry, even one considers a system with the isotropic
parameters for the three components. Therefore, the phenomena discussed
in the present work are completely different from those in magnetic or
multi-ferroic systems.

\begin{quote}
C2: Indeed, both referees, in particular B, point out that the LG
expression is introduced out of thin air, without relating it to any
specific system (even after the second submission!). In that context, I
find the reply of the authors to Referee A unacceptable : ``However, we
notice that the previous papers treated primarily the isotropic case, a
situation very rare, if not impossible, in reality, and did not discuss
whether the TRSB superconductivity exists in a more general situation.''
The present article does not offer any relation to a real physical
system, except for announcing right at the beginning that it has to do
with our understanding of the birth of the universe (!)``. Furthermore,
referee A commented that the fact that such superconductors cannot be
categorized neither into type I nor into type II has been already
discussed in the literature. Instead of properly answer him, the authors
chose to declare ``the failure of categorization of TRSB
superconductivity into type I and II has not yet been discussed in
literature so far, oppositely to the comment from this Referee.'' This
is simply not true. Typing ``Type 1.5 superconductivity'' in Google
yielded 428,000 entries. It might be that the reason is different than
in the present analysis, but still one expects the authors to comment on
the difference.
\end{quote}

A2: The first part of this comment is responded in the above reply A1.

We mentioned the importance of the symmetry breaking in various contexts
of physics in the introduction. But it is clear that we have not claimed
any relation between our GL free-energy and universe.

Now let us respond to the comment by this Referee on the categorization
of type I and type II superconductivity. The first, and most important,
point is the ``fact'' that there is no discussion in literatures on
categorization of type I or type II of the TRSB superconductivity or
failure of it. Secondly, our statement is that the GL number
kappa=lambda/xi cannot be used to categorize the TRSB superconductivity
(type I or type II). This statement is derived from our analytic results
on coherence length, nucleation field and thermodynamic critical field.
We have not claimed that TRSB superconductivity is a new type one beyond
type I and type II. Thirdly, type 1.5 superconductivity mentioned by
this Referee is a topic of big controversy. While this issue is beyond
the scope of the present work and a full discussion takes another
separated paper, we include some inputs here just for the merit of
communication with this Referee:

The so-called type-1.5 I superconductivity has its root in a paper by
Babaev and Speight (Ref. 17) for a two-component system (where TRSB
superconductivity impossible) without inter-component coupling. There
are papers and discussions on the possibility of type-1.5 in journals
and arXiv, both positive and negative to the idea, and thus very
controversial. If one calculates carefully a two-component system with
finite coupling, it becomes clear that there is only one divergent
coherence length, in sharp contrast with TRSB case. When one
concentrates on temperature regime close to Tc (in our opinion one
should since GL is the starting point), this divergent coherence length
dominates over any other non-divergent length scale(s), the system is
not different from a single component superconductivity, and especially
the system is either type I or type II.

Lastly, we wish to make it clear that we responded to Referee A with
detailed discussions on the issue of categorization of superconductivity
(see R.1.4 \& R.1.5 in the first report), which we think properly
enough, before mentioning briefly the absence of similar discussion in
literature related to TRSB state.

\begin{quote}
C3: Whereas the authors address the genuine comments of Referee A rather
aggressively (for no apparent reason), they treat Referee B extremely
politely, but nonetheless, fail to supply the required clarifications.
For example, referee B pointed out the possibility of a magnetic order
in the pnictides, which is not included in the GL expression (1) of the
manuscript. As the only physical system they mention is the iron
pnictides, I find the authors' reply ``Meanwhile, we would like to
notice that there is no established microscopic theory, and no confirmed
experimental evidence either, for the coexistence of magnetism and
superconductivity in these materials. Therefore, it is natural not to
include a magnetic part in the GL free-energy functional. The
time-reversal symmetry breaking discussed here is purely caused by the
frustrated couplings among superconducting components. It is unique
since it is not accompanied by an internal magnetic field like a triplet
superconductor.'' ) very unsatisfactory. What is then the reason for the
frustrated couplings? in which physical system can they be realized? (I
am echoing the comments of referee B.)
\end{quote}

A3: We responded to the comments by Referee A directly, but not
``aggressively''. Our responses would be better if softer, but we
believe that as a scientific communication, the most important issue is
the ``fact''.

As for the absence of competing magnetic order in the GL expression,
basically we have already communicated with Referee B in the last
rounds. We wish to emphasize here that in most known iron-pnicitide
superconductors, superconductivity appears after the ``long-range
order'' of antiferromagnetism is suppressed (no ferromagnetic
fluctuation was reported). Since GL theory treats the long-range orders,
antiferromagnetism should be dropped as a competing order. As discussed
in A1, there is no microscopic derivation of the frustrated
inter-component couplings at this moment, and we chose to work on the
general condition for the TRSB superconductivity. It is a hope that
iron-pnictide superconductors can realize the frustrated situation in
certain doping regime.

\begin{quote}
C4: I also find their comment (both in the manuscript and in their reply
to referee B) ``and third we have made sure that including more terms
-in the GL expression- will not change the novel physics addressed
here.'' What do they mean? which additional terms did they check?
Equation (1) of the manuscript includes just one possible combination of
the fourth-order terms, out of many more (see Ref. 5 in this context),
for example, terms that couple the three order parameters.
\end{quote}

A4: We thank this Referee for this question. The most important terms
which cause the TRSB state is the Josephson-like couplings included in
Eq.(1). Couplings we have checked include: (1) cross quadratic terms
with gradients of order parameters, (2) all quartic terms including
\(| \psi_j|^2| \psi_k|^2\) and \$ \psi\_j \psi\^{}*\_k\textbar{}
\psi\_l\textbar{}\^{}2\$. Including these details will not change the
main conclusion of the present work, namely the stability of the TRSB
state and the novel Josephson effect, and the process to arrive at these
results, but will modify the expressions of some results. In order to
respond to this comment we modify our discussions below Eq(1).

\begin{quote}
C5: In summary, the paper presents an interesting analysis of the GL
expression which includes three superconducting order-parameters with
competing interactions. The authors are advised to revise their work
along the lines proposed by the previous referees. I think that they
should relate their work to actual systems, and also to similar problems
appearing in the context of magnetism (i.e.~coupled x-y models, and
maybe even multiferroics).
\end{quote}

A5: We thank this Referee for the positive evaluation on our present
work. As for the relation with magnetism, please refer to our reply A1.

\hypertarget{prl-round-4}{%
\subsection{PRL Round 4}\label{prl-round-4}}

\hypertarget{referee-c}{%
\subsubsection{Referee C}\label{referee-c}}

I join referee B in concluding that the authors seem to be unwilling to
modify their paper.

Point A--the validity of the LG theory: It seems that the authors did
not understand my comment. Using the LG functional as a phenomenological
theory is quite ubiquitous, and many times very fruitful. However, the
claim or the hint that the specific functional they choose to study
describes the iron-pnictide superconductors needs some substantiation,
either by a careful investigation of symmetries in the pnictides, or by
comparing the outcome of the theory with experiments (or both). None of
this is presented in this paper.

Point B --type I or type II categorization: The authors may have
discussed this topic in length in their reply to referee A, but there is
no discussion of this point in the paper, and no comparison with other
systems (of different symmetries) in which such a categorization is not
possible as well.

Point C--The authors' reply is rather confusing. On one hand, they keep
alluding to the pnictides as the systems they wish to describe, and on
the other they express their ``hope'' that the theory will be realized
by those materials. In case they believe in the first option, then one
would like to have some discussion of the relation between the theory
and the experimental observations. In the other case, they still have to
convince the reader why do they analyze this specific LG functional, and
what should be the properties of the systems (as revealed in
experiments) for which it will be relevant.

Let me also add in passing that there is no place for the sentence
mentioning the ``birth of the universe'' in a paper devoted to a topic
in condensed-matter theory.

\hypertarget{referee-b}{%
\subsubsection{Referee B}\label{referee-b}}

The authors have convinced me of their arguments and I conclude that
apart from the title and modifications to abstract the paper is almost
ready to be moved over to PRB for consideration. That could be the right
venue for it. It is not as accessible to a broad audience as the authors
think it is, and therefore PRL is not the the right journal for it.

The title should be shortened, preferably, and stated as follows: ``An
Analysis of Stability and Josephson effect in time reversal symmetry
broken superconductivity''.

I do not think there is anything novel about the Josephson effect seen
in a superconductor constriction superconductor junction. Yes when used
as a probe it can provide a signature of the difference in TRSB states.
However this is not unique to this work. The Josephson effect has been
used as a probe for a long time now.

\hypertarget{prb-round-1}{%
\subsection{PRB Round 1}\label{prb-round-1}}

Dear Editor Physical Review B:

We now transfer our paper entitled ``Stability and novel Josephson
effect of time-reversal-symmetry-broken superconductivity induced by
frustrated inter-component couplings'' with code LC13464 to Physical
Review B for your consideration.

Our paper submitted to Physical Review Letter took a long review
process. We had a hard time to understand the comments from Referee B.
At the first round, this referee evaluated our work very positively, and
in the second round s/he raised a question on the expression for the
critical current for the constriction junction. After all the
discussions of four rounds, we finally convinced this referee that our
result is correct. But finally this referee turned our submission down.
To us the comments from this referee seem inconsistent.

Since the reviewing time is too long as a Letter, we choose to change it
to Physical Review B, and seek a reasonably smooth treatment.

Thank you very much for your consideration.

Sincerely, Xiao Hu and Zhi Wang

\hypertarget{reply-to-the-second-report-of-referee-c}{%
\subsubsection{Reply to the second report of Referee
C:}\label{reply-to-the-second-report-of-referee-c}}

\begin{quote}
Coomment-A: the validity of the LG theory: It seems that the authors did
not understand my comment. Using the LG functional as a phenomenological
theory is quite ubiquitous, and many times very fruitful. However, the
claim or the hint that the specific functional they choose to study
describes the iron-pnictide superconductors needs some substantiation,
either by a careful investigation of symmetries in the pnictides, or by
comparing the outcome of the theory with experiments (or both). None of
this is presented in this paper.
\end{quote}

Reply A: Although the iron-pnictide superconductors with more than three
components, and reportedly of frustrated inter-component interactions
are the candidate materials for the phenomena addressed in the present
work to occur, no microscopic theory for the mechanism for them is
available now. A derivation of GL theory is thus impossible at the
present stage. We have to admit that we cannot provide a substantiation
for our work in this sense.

The functional we adopt is not specific for a multi-component
superconductor as can be found in literatures for two-component
superconductors. For a clean superconductor, one can derive the GL
functional following the work by Zhitomirsky and Dao, PRB, 69 (2004)
054508, which was discussed in the review paper by Gurevich, Physica C
456 (2007) 160 (Ref.14 in the present version of manuscript). In order
to respond to this comment, we add the paper by Zhitomirsky and Dao to
our reference list.

In the present work, we derive for the first time the general stability
condition for the TRSB superconductivity, and then predict
``analytically'' several unique properties for this novel state. Up to
this moment, there is no direct experimental observation. But we believe
that this does not degrade the importance of our theory.

\begin{quote}
Comment-B: type I or type II categorization: The authors may have
discussed this topic in length in their reply to referee A, but there is
no discussion of this point in the paper, and no comparison with other
systems (of different symmetries) in which such a categorization is not
possible as well.
\end{quote}

Reply-B: We calculated analytically thermodynamic field, thermodynamic
field, the penetration depth, and two ``divergent'' coherence lengths
for this TRSB superconductivity. One then finds the ratio between the
nucleation magnetic field and the thermodynamic field does not equal to
\sqrt{2}\kappa with \kappa=\lambda/\xi, the Ginzburg-Landau number.
Therefore, the conventional classification of superconductor by
\kappa does not work for the present TRSB superconductivity. The
statement was clearly put in the manuscript. To be short, the
categorization of type I and type II does not fail, but the
Ginzburg-Landau number is not the right index for categorization in the
present TRSB state.

We guess the Referee is requesting us to touch on the so-called type-1.5
superconductors. However, there are several important differences
between those works and the present one as follows. First, possible
type-1.5 superconductors were discussed for two-component
superconductors; Here we reveal the TRSB state in three-component
superconductors, which is impossible in two-component superconductors.
Secondly, in all two-component superconductors there is only one
divergent coherence length; In the present TRSB state, there are two
divergent coherence lengths. Thirdly, type-1.5 superconductors (if any)
are different from type II superconductors at H\_c1; We discuss the
behavior of the superconductor around the magnetic field where
superconductivity nucleates when lowering the field, namely around H\_c
for type I, and around H\_c2 for type II superconductors. We did have
some discussions with Referees A on this point and tried to make the
situation clear, namely the present work has no direct relation with
those for type-1.5 superconductor.

In response to this comment, we add discussions in the last part of the
manuscript.

\begin{quote}
Comment-C: The authors' reply is rather confusing. On one hand, they
keep alluding to the pnictides as the systems they wish to describe, and
on the other they express their hope" that the theory will be realized
by those materials. In case they believe in the first option, then one
would like to have some discussion of the relation between the theory
and the experimental observations. In the other case, they still have to
convince the reader why do they analyze this specific LG functional, and
what should be the properties of the systems (as revealed in
experiments) for which it will be relevant.
\end{quote}

Reply C: This comment is basically same as Comment-A. We repeat our
points: the GL functional is not specific. It can be derived from BCS
theory for a three-component superconductor. We then formulate the
conditions for TRSB state, and reveal its novel features. We propose a
constriction junction setup to test experimentally the addressed TRSB
superconductivity.

\begin{quote}
Comment D: Let me also add in passing that there is no place for the
sentence mentioning the ``birth of the universe'' in a paper devoted to
a topic in condensed-matter theory.
\end{quote}

Reply D: We follow this suggestion and remove this part.

\hypertarget{reply-to-the-fourth-report-of-referee-b}{%
\subsubsection{Reply to the fourth report of Referee
B:}\label{reply-to-the-fourth-report-of-referee-b}}

\begin{quote}
Comment 1: The authors have convinced me of their arguments and I
conclude that apart from the title and modifications to abstract the
paper is almost ready to be moved over to PRB for consideration. That
could be the right venue for it. It is not as accessible to a broad
audience as the authors think it is, and therefore PRL is not the the
right journal for it.
\end{quote}

Reply-1: We are glad to know that this Referee finally gets our point on
the constriction junction. But then we don't understand why s/he changes
the evaluation on our contribution, after all these communications.

\begin{quote}
Comment 2: The title should be shortened, preferably, and stated as
follows: ``An Analysis of Stability and Josephson effect in time
reversal symmetry broken superconductivity''.
\end{quote}

Reply-2: As discussed in the manuscript, the present TRSB
superconductivity caused by frustrated inter-component superconductor is
quite different from other TRSB ones, such as a spin-triplet case.
Therefore, to convey the important ingredients of the present study, it
is better to include the terms ``frustrated inter-component coupling''
in the title. The title suggested by this Referee may leave an
impression to readers that we are talking about triplet superconductors.

\begin{quote}
Comment 3: I do not think there is anything novel about the Josephson
effect seen in a superconductor constriction superconductor junction.
Yes when used as a probe it can provide a signature of the difference in
TRSB states. However this is not unique to this work. The Josephson
effect has been used as a probe for a long time now.
\end{quote}

Reply 3: The TRSB superconductivity caused by frustrated inter-component
coupling has not been explored clearly so far and is thus novel. The
constriction junction itself is not new, but it can provide unambiguous
evidence the novel TRSB superconductivity, and thus deserves to be
highlighted in our opinion.

\hypertarget{prb-round-2}{%
\subsection{PRB Round 2}\label{prb-round-2}}

\hypertarget{third-report-of-referee-c-lc13464bhu}{%
\subsubsection{Third Report of Referee C --
LC13464B/Hu}\label{third-report-of-referee-c-lc13464bhu}}

The present version of the manuscript requires significant modifications
before it can be considered for publication in Phys. Rev.~B.

First, the introduction section is far too terse and sketchy, to the
point that makes the topic inaccessible to most of the journal readers.
Two points in particular should be expounded upon. a. The new aspect of
this article is the extension to anisotropic couplings between the three
components of the order parameter, whereas previous works dealt with
symmetric couplings alone. It would be helpful to learn the authors'
reasons for this extension, and why do they think that their scenario
describes better the iron-pnictides than the isotropic-coupling
scenario. b. The proposed experiment--what are the features that one
should look for in such an experiment? the authors confine themselves to
``suppressed significantly" which is rather vague.

Section III, devoted to the (rather standard) calculation of the
coherence length, penetration length, and critical fields is partially
based on the \{\em isotropic\} scenario. What are then the new features
here as compared to Refs. 5-7?

Section IV, which describes the proposed experiment presents an
expression for the Josephson current, that includes quite a number of
parameters. Nonetheless, the authors expect it to provide evidence for
the existence of the anisotropic state. In particular, a detailed
comparison between the Josephson current of the isotropic scenario and
the one in the present manuscript should be carried out.

Finally, the number of times the word
\texttt{novel\textquotesingle{}\ or}brand new'' is scattered around
seems to be exaggerated. I have counted six of them in the first page
alone (and there are many others later on), including the dichotomic
sentence
\texttt{Interests\ in\ this\ novel\ phenomenon\ are\ renewed...."\ The\ same\ comment\ pertains\ to\ the\ extensive\ use\ of\ the\ phrase}smoking-gun
evidence"\ldots{}

\hypertarget{report-of-the-fourth-referee-lc13464bhu}{%
\subsubsection{Report of the Fourth Referee --
LC13464B/Hu}\label{report-of-the-fourth-referee-lc13464bhu}}

Having gone through the manuscript, now submitted to PRB, and about 20
pages of reports and reply of the previous PRL submission, I think that
the manuscript is generally suitable for PRB. However it needs some
modifications. The present manuscript is written in a very dense way
and, thus , in some places is difficult to understand. I give examples:

\begin{itemize}
\item
  In the title, ``inter-component couplings'' is not very well defined
  (what is coupled to what?). I suggest to extend ``superconductivity''
  to something like ``multiband superconductivity'' , ``multicomponent
  superconductivity'', ``three-component superconductivity'' etc. to
  make this clearer.
\item
  In the abstract and also in the introduction the authors should
  sharpen the expression ``frustrated''. In the present manuscript this
  isn't clear before section II.
\item
  ``GL number'' and ``GL parameter'' are exchanged randomly throughout
  the manuscript. ``GL number'' may be confused with ``Ginzburg
  number''. I thus suggest to always use ``GL parameter''.
\item
  The expression ``tunneling'' for the constriction-type Josephson
  junction seems to be somewhat misleading. This could be rephrased to
  ``weak coupling''. E. g. Section IV would read ``Weak coupling of two
  TRSB states'' in this case.
\item
  In some places one is referred to literature to understand the
  approach taken. Examples are Sylvester's criterion in section II and
  ``the idea developed for conventional single component
  superconductors'' in section IV. Some additional explanations would be
  very helpful here. Also, some sentences like the one above equation
  (9) sound very cryptic.
\item
  Reversely, the discussion and summary section is very short and, e.
  g., the message that the GL number is insufficient for categorizing
  the TRSB is repeated within two paragraphs. I suggest to extend the
  discussion, perhaps as an extra chapter.
\item
  As a minor detail I suggest to make one of the lines in Fig. 2 dashed
  or dotted. The lines look very similar when printed without colors.
\item
  The English needs some improvement. Also there is a typo
  ``superconductivity'' 4 lines above eq. (4).
\item
  In terms of the issue of Type 1.5 superconductivity raised by the
  previous referees -- this topic is under quite some dispute, to say
  the least, and I completely agree with the authors to not include this
  in the present manuscript.
\end{itemize}

In summary, I think that the authors should extend the manuscript by one
page or so. Then, in my opinion it makes a good and valid contribution
to Phys. Rev.~B.

\hypertarget{second-report-of-the-fourth-referee-lc13464bhu}{%
\subsubsection{Second Report of the Fourth Referee --
LC13464B/Hu}\label{second-report-of-the-fourth-referee-lc13464bhu}}

In their revised manuscript the authors included all my suggestions.
Regarding the third report of Referee C I feel that his fourth comment
on the excessive use of novel, brand new, smoking gun, etc., should be
taken into account. In their reply the authors say that they will change
the expressions when necessary -- I think it is necessary.

After these changes have been made, in my opinion the manuscript is
ready for publication.

\hypertarget{epl-report}{%
\subsection{EPL REPORT}\label{epl-report}}

The authors discuss a multi-band superconductor consisting of three
bands. They show that for repulsive inter-band Cooper pair scattering
this system may form a time-reversal symmetry breaking state. They give
a detailed discussion of a generalized Ginzburg-Landau model and discuss
a few consequences. The analysis is sound. Nevertheless, there is not
much new relevant input is given by the paper. Moreover most results are
neither experimentally testable nor relevant. I also do not agree with
their statement that we cannot distinguish type I and type II
superconductivity in this case. When ever Hn (Hc2) is higher than the
thermodynamic critical field we consider the superconductor to be of
type II. Also the reduction of the critical current in the Josephson
junction due to phase frustrated Josephson coupling is in principle
correct. However, the authors give no clue in which way you would notice
experimentally such reduction.

In my opinion this manuscript does not qualify for publication in EPL. I
rather suggest to transfer it to another journal such as EPJB possibly
after some expansion in some parts. It is important to give the
manuscript more physical relevance to avoid the simple mathematical
discussion of a simple model.

    \hypertarget{detecting-majorana-fermions-by-nonlocal-entanglement-between-quantumdots}{%
\section{Detecting Majorana fermions by nonlocal entanglement between
quantumdots}\label{detecting-majorana-fermions-by-nonlocal-entanglement-between-quantumdots}}

\hypertarget{prl-round-1}{%
\subsection{PRL Round 1}\label{prl-round-1}}

\hypertarget{report-of-referee-a-ll13514wang}{%
\subsubsection{Report of Referee A --
LL13514/Wang}\label{report-of-referee-a-ll13514wang}}

The subject of Majorana states and detection of nonlocal entanglement is
very hot and many papers (especially, theoretical ones) are devoted to
it. The present manuscript belongs to the same category. Unfortunately,
present referee did not find the proposal of Z.Wang and X.Hu to be
(even) semi-convincing. General impression after reading the manuscript
is as follows: the properties of the model 4x4 Hamiltonian studied are
hardly of interest for a real Physica system proposed.

Below I list few specific questions and comments:

\begin{enumerate}
\def\labelenumi{\arabic{enumi}.}
\item
  The Hamiltonian (5) does not conserve number of electrons in the whole
  system. Still the authors use the basis of 4 eigen-states with fixed
  full number of electrons equal to 2N+2. How is it possible ?
\item
  It is quite unclear, in which respect the 4x4 Hamiltonian (8) is
  related to the presence of Majorana state in a superconductor. It
  seems that the only important ingredient used is just the Coulomb
  blockade. It is useful to mention here that after reduction to 4-state
  space any information about nonlocality of Majorana state is lost.
\item
  The paper does not contain any realistic estimates of the scales of
  physical quantities involved (energies, temperatures, etc.) which
  would make possible to access how realistic (at least in principle) is
  the proposal discussed.
\item
  It is quite unclear which specific type of measurement should be used
  to detect ``entanglement'' expressed in terms of the difference
  P\_1-P\_2. The only statement related to this issue is in the last
  sentence of the text before ``summary'', and it does not seem to be
  convincing.
\end{enumerate}

\hypertarget{report-of-referee-b-ll13514wang}{%
\subsubsection{Report of Referee B --
LL13514/Wang}\label{report-of-referee-b-ll13514wang}}

The manuscript LL13514 analyses the entanglement between two dots
connected by a topological superconductor. Building on the work by Liang
Fu {[}20{]} and Klensberg {[}21{]}, the authors notice that the
occupation of one dot depends on the occupation of the other dot,
provided the superconductor has a significant charging energy and
Majorana end states. They suggest measurement of this could be used as a
detection of Majorana fermions.

The paper is very clearly written and straightforward. Still I am not
convinced that it has enough novelty with respect to Refs {[}20,21{]}
and bring sufficient impact in the field to justify a publication in
PRL.

Besides, I have one concern. In order to probe entanglement, one has to
be very careful in defining the measuring protocol. I find that the
authors should first define more clearly the protocol by which
experimentalists will have to measure the occupancy of the one dot
knowing that the other dot is occupied. Second, a more detailed analysis
of the experimental feasibility is required. In practice, the last
paragraph before conclusion, {[}Lastly we discuss about
possible\ldots{}.radio-frequency single-electron transistor{]}, should
be strengthened to my opinion.

In conclusion, I would like to reconsider the paper after authors will
have addressed the previous issues.

\hypertarget{remarks-intended-solely-for-the-editor}{%
\subsubsection{Remarks intended solely for the
editor:}\label{remarks-intended-solely-for-the-editor}}

Dear Dr.~Samindranath Mitra Associate Editor Physical Review Letters

Thank you for your email contact on our paper LL13514 as of Jan.~19,
2012.

We read carefully the comments from the Referees. We find that both
Referees evaluate positively the importance of our work on detection of
Majorana fermions. Referee A asked about the formalism of our study,
which Referee B took as very clear. But Referee B raised the question
about the novelty of of our proposal as compared with two previous
studies.

We feel that we can overcome the criticisms as documented in the Reply.
We then revise our manuscript accordingly, and make it clear that our
proposal is a non-invasive method, which is complimentary to all the
methods proposed so far. Considering the current status of intensive
experimental searching for Majorana fermions and that no conclusive
confirmation has been reported, our proposal should be important.

We sincerely hope that our new manuscript can be accepted for
publication in Physical Review Letters.

Thank you very much in advance for your consideration.

Yours sincerely, Xiao Hu and Zhi Wang

\hypertarget{reply-to-referee-a}{%
\subsubsection{Reply to Referee A:}\label{reply-to-referee-a}}

\begin{quote}
Comment1: The subject of Majorana states and detection of nonlocal
entanglement is very hot and many papers (especially, theoretical ones)
are devoted to it. The present manuscript belongs to the same category.
\end{quote}

Reply1: First, we thank this Referee for his careful reading and
positive evaluation on the general interest of our paper.

\begin{quote}
Comment2: Unfortunately, present referee did not find the proposal of
Z.Wang and X.Hu to be (even) semi-convincing. General impression after
reading the manuscript is as follows: the properties of the model 4x4
Hamiltonian studied are hardly of interest for a real Physica system
proposed.
\end{quote}

Reply2: We cannot agree with the Referee on this point. The 4x4
Hamiltonian in the present work is derived based on the reduction of
Hilbert space, which is powerful in grasping the low temperature physics
of the system involving Majorana fermions.

First, we wish to point out that the procedure of reduction of Hilbert
space itself is now a standard one, well established by research
activities in the field of Josephson charge qubit. For example, in the
review paper for Josephson-junction qubit (Y. Makhlin, G. Sch"\{o\}n, A.
Shnirman, Rev.~Mod. Phys. \textbf{73}, 357 (2001)), starting from page
359, it was explicitly shown that the dimension of the Hilbert space can
be reduced from infinity to two in a Josephson charge qubit, leading to
a 2x2 model Hamiltonian which grasps the physics of the system at
sufficiently low temperature. The charge qubit has already been realized
experimentally, and the experimental results confirmed the validity of
the procedure of Hilbert-space reduction, as well as the matrix model.

In topological superconductors with Majorana fermions, there is a
two-fold degenerate ground state, corresponding to the even and odd
fermionic parity of the system. In presence of charging effect
associated with the mesoscopic size of the superconductor, while the
degeneracy is lifted, a quasiparticle state exists with energy well
below the superconductivity gap. This feature makes the reduction of
Hilbert space possible in topological superconductors. As a matter of
fact, this approach has been adopted in studies of systems involving
Majorana fermions. For example, in Refs. 20 and 21 the authors derived a
reduced 2x2 Hamiltonian for the topological superconductor, for the two
states describing by two Majorana fermions, or equivalently two states
with even or odd parities.

Our model Hamiltonian is based on the root exactly same as their works,
with the Hilbert space doubled due to the two quantum dots connected in
a tunneling way to the topological superconductor. As detailed in
manuscript, and also in the following replies, the present setup can
provide a new way to detect Majorana states, one of the frontiers of
this field as pointed out by this Referee.

\begin{quote}
Comment3: The Hamiltonian (5) does not conserve number of electrons in
the whole system. Still the authors use the basis of 4 eigen-states with
fixed full number of electrons equal to 2N+2. How is it possible ?
\end{quote}

Reply3: First, we notice that the Hamiltonian (5) in our paper conserves
number of electrons, which includes the topological superconductor and
two quantum dots. The important point here is to keep in mind that the
phase factor \(e^{\pm i\phi/2}\) in \(H_T\) increases or decreases the
charge number of the superconductor by one electron unit.

In conventional mean-field theory for bulk superconductivity, the BCS
Hamiltonian does not conserve number of electrons, and the BCS ground
state is a superposition of states with different number of electrons.
However, when dealing with superconducting qubit systems, the
superconductor is (at least semi) isolated, and the description of fixed
electron number has been used, in good agreement with experimental
observations. In our setup, a topological superconductor with mesoscopic
size is considered in connection with two quantum dots. Therefore, the
whole system conserves the number of electrons. At low temperature where
the quasiparticle states with energy larger than the superconductivity
gap can be neglected and the 4x4 Hamiltonian is valid, the number of
electrons is clearly conserved as well.

\begin{quote}
Comment4: It is quite unclear, in which respect the 4x4 Hamiltonian (8)
is related to the presence of Majorana state in a superconductor. It
seems that the only important ingredient used is just the Coulomb
blockade. It is useful to mention here that after reduction to 4-state
space any information about nonlocality of Majorana state is lost.
\end{quote}

Reply4: The presence of Majorana fermions is crucial for the description
of the system with a 4x4 Hamiltonian. Without the Majorana fermions,
electrons on the quantum dots would have to tunnel into quasiparticle
states of the superconductor, involving physics related to states in a
continuum spectrum, which is clearly impossible to be described by a 4x4
Hamiltonian. In contrast, the charging energy is important for detection
of the QD entanglement in our work, but unnecessary for the reduction of
Hilbert space, as revealed by the result
\textbf{without charging energy} shown in Fig. 3 of our paper. The
nonlocality of the electronic state in the topological superconductor
formed by Majorana fermions lays the ground for our 4x4 Hamiltonian, and
is thus inherited in the physics addressed in the present manuscript.

\begin{quote}
Comment5: The paper does not contain any realistic estimates of the
scales of physical quantities involved (energies, temperatures, etc.)
which would make possible to access how realistic (at least in
principle) is the proposal discussed.
\end{quote}

Reply5: We thank the Referee for this suggestion and add an explicit
discussion for the scales of physical quantities to the manuscript. The
superconducting energy gap is in the order of 10K, and the charging
energy and tunneling energy should be smaller than the energy gap by one
order in magnitude, and thus should be around 1K. In order to fixating
on ground state, the operating temperature should be around 100mK, which
is achievable with the-state-of-art technology.

\begin{quote}
Comment6: It is quite unclear which specific type of measurement should
be used to detect ``entanglement'' expressed in terms of the difference
\(P_1-P_2\). The only statement related to this issue is in the last
sentence of the text before ``summary'', and it does not seem to be
convincing.
\end{quote}

Reply6: We thank the Referee for pointing out the importance of possible
measuring methods. We extend our discussions on the single charge
transistor (SET) technique useful for our proposal. SET is a well
developed technique for measuring electron occupation in targeting
system, basing on a simple mechanism. It is a small circuit with a
mesoscopic metallic island, whose conductance is extremely sensitive to
nearby charges. When putting close to a QD, the conductance measured
from the SET will be different for different electron occupation on the
QD. The SET is a non-invasive way to measure the QD, yet has extremely
high sensitivity and small response time. In a well designed
radio-frequency SET, it has been shown that a low charge noise and a
microsecond response speed can be achieved. With this SET technology, we
can detect simultaneously the electron occupancy on the two QDs, and
evaluate the conditional occupancy probabilities \(P_1\) and \(P_2\) by
repeating the measurements, thus determine the entanglement between QDs
in our proposal.

\hypertarget{reply-to-referee-b}{%
\subsubsection{Reply to Referee B:}\label{reply-to-referee-b}}

\begin{quote}
Comment1: The manuscript LL13514 analyses the entanglement between two
dots connected by a topological superconductor. Building on the work by
Liang Fu {[}20{]} and Klensberg {[}21{]}, the authors notice that the
occupation of one dot depends on the occupation of the other dot,
provided the superconductor has a significant charging energy and
Majorana end states. They suggest measurement of this could be used as a
detection of Majorana fermions.
\end{quote}

Reply1: First, we thank this Referee for the careful reading.

\begin{quote}
Comment2: The paper is very clearly written and straightforward. Still I
am not convinced that it has enough novelty with respect to Refs
{[}20,21{]} and bring sufficient impact in the field to justify a
publication in PRL.
\end{quote}

Reply2: We thank the Referee for his assent to our paper. We clarify the
novelty and impact of our paper in more detail as follows.

First, our proposal for detecting the existence of Majorana fermions is
based on the quantum correlation, or entanglement, between the two QDs
close to the two ends of a topological superconductor-wire. To the best
of our knowledge, it is the first work which differs from all proposals
thus far in literatures based on transport properties, such as the
4\(\pi\) period Josephson current, the Andreev reflection, and the
anomalous conductivity. As far as transports are involved, one should
always be very careful since the signal of current might be influenced
by other factors, such as the interface and/or contact condition.
Moreover, the current flowing through the topological superconductor
might disturb, or even destroy the Majorana states. In contrast, our
proposal is based on weak tunneling, which does not destroy the
topological protection of the Majorana fermions. In this sense, our
method is a non-invasive one, and forms another category of detection
methods of Majorana fermions. Considering the current status of
intensive experimental searching for Majorana fermions and that no
conclusive confirmation has been reported, our proposal, which is at
least complimentary to other proposals raised so far, is expected to be
important.

Additionally, since the QD entanglement is induced by the entanglement
of Majorana fermions, our proposal, if successful, not only proves the
existence of Majorana fermions, but also their nature of quantum
entanglement. Our non-invasive detection on Majorana fermions may also
open the door for the quantum non-demolition measurement (QND) of the
topological qubit in the future.

Now let us discuss the novelty of our work more specifically with
respect to Refs 20 and 21. In Ref. 20, Fu studied the transport
properties of a topological superconductor in presence of charging
energy. He revealed explicitly that Majorana fermions will induce novel
transport property when charging effect is taken into account.
Therefore, his work belongs to the category involving current signal,
which is different from ours as discussed above. Ref. 21 by Flensberg
was the first paper which addressed the interaction between Majorana
fermions and quantum dots. However, his work is irrelevant to detection
of Majorana fermions, which is the key result of our paper. We benefit
in formulation from these two nice works, but the physics addressed here
is different from theirs.

In short, comparing with previous works, our proposal provides a
complimentary, and better in our opinion, solution for detecting
Majorana fermions, which has many merits and is experimentally possible.
We believe our paper has sufficient novelty and impact such that it
deserves to be published in PRL.

\begin{quote}
Comment3: Besides, I have one concern. In order to probe entanglement,
one has to be very careful in defining the measuring protocol. I find
that the authors should first define more clearly the protocol by which
experimentalists will have to measure the occupancy of the one dot
knowing that the other dot is occupied. Second, a more detailed analysis
of the experimental feasibility is required. In practice, the last
paragraph before conclusion, {[}Lastly we discuss about
possible.radio-frequency single-electron transistor{]}, should be
strengthened to my opinion.
\end{quote}

Reply3: First, we thank the Referee for suggesting us to strengthen the
discussion on the experimental feasibility. We make a more detailed
discussion about the single charge transistor (SET) technique useful for
our proposal. It is a small circuit with a mesoscopic metallic island.
The conductance of the circuit is sensitive to nearby charges due to
Coulomb interaction, thus able to detect the electron occupation of a
targeting system. It has been reported experimentally that a well
designed radio-frequency SET can detect QD occupation with low charge
noise within microseconds. We believe this technology is good enough for
the measurement of electron occupation on QDs, which we need for
determining the entanglement in our proposal.

Since the topological superconductor is not measured directly in our
proposal, the two QDs are actually in a mixed state. Because
entanglement in a mixed state is quite complex, and moreover it is
unnecessary to our purpose of detecting Majorana fermions, we simply
defined the QD entanglement in our paper as the difference in the
occupancy probabilities in one QD when the other QD is occupied or
vacant. The protocol for measuring the entanglement of QDs is thus
simple and straightforward, namely we detect simultaneously the electron
occupancy on the two QDs by the SET, and evaluate the conditional
occupancy probability \(P_1\) and \(P_2\) by repeating the same
measurements, and use \(P_2-P_1\) as a measure of the QD entanglement.
Our interest is to construct a signal for detecting Majorana fermions,
which can be achieved without addressing the Bell inequality which
characterizes fully quantum entanglement. In our context, the word
`entanglement' is equivalent to `quantum correlation'. If necessary, we
are willing to change the nomenclature.

\begin{quote}
Comment4: In conclusion, I would like to reconsider the paper after
authors will have addressed the previous issues.
\end{quote}

Reply4: We thank the Referee for his/her careful reading and valuable
comments. We sincerely hope that our replies and corresponding revisions
can accommodate the comments, and appreciate very much the
recommendation from this Referee.

\hypertarget{summary-of-changes}{%
\subsubsection{Summary of changes:}\label{summary-of-changes}}

\begin{enumerate}
\def\labelenumi{(\arabic{enumi})}
\item
  On page 1, left column, the beginning of the third paragraph, ``A
  number of proposals have been suggested to detect MFs in
  superconducting systems, mostly by tunneling experiments
  {[}7,14-19{]}.'' is changed to ``A number of proposals have been
  suggested to detect MFs in superconducting systems {[}7,14-20{]}, such
  as Andreev effect, Josephson effect, and anomalous electron current.''
\item
  On page 1, right column, the end of the first paragraph, we add ``Our
  approach does not involves electric current, which differs from all
  the methods proposed so far for detecting MFs, and can be conducted in
  a non-invasive way experimentally.''
\item
  On page 2, left column, below eq. (4),
\end{enumerate}

``In the following, we consider the superconducting energy gap to be the
largest energy scale and set it as infinity, thus the superconductor is
in its ground state and the superconducting quasiparticles are
irrelevant, which is a good approximation as far as low temperature is
concerned.''

is changed to

``The energy gap of the topological superconductor has been
estimated{[}7{]} to be around \(\Delta \sim 15K\). Here we take the
superconducting energy gap as the largest energy scale, and concentrate
on low-temperature physics where quasiparticles in the continuum
spectrum above the energy gap are irrelevant. The charging energy
\({e^2}/{C}\) and the tunneling energy \(T_j\) are tuned both smaller
than \(\Delta\), say around 1K, which is one order smaller than the
energy gap, and can always be achieved by increasing the size of the
superconductor and the separation between QDs and superconductor. In
order to keep the system in the ground state, we set the operating
temperature around 100mK, a range well accessable experimentally. For
simplicity, the energy gap is taken as infinity in what follows.''

\begin{enumerate}
\def\labelenumi{(\arabic{enumi})}
\setcounter{enumi}{3}
\tightlist
\item
  On page 4, right column, the paragraph before summary:
\end{enumerate}

``Lastly we discuss about possible experimental implementations of our
idea. Our method does not require any measurement on the state of MFs
and the topological superconductor. All we need is the measurement of
the electron occupation on QDs. Charge sensing on QD is a well developed
area {[}23{]}, and the measurement of charge occupation on QD has
already been achieved with integrated radio-frequency single-electron
transistor {[}24{]}.'' is changed to ``Lastly we discuss about
experimental implementation of our idea. Our method does not require a
direct measurement on the state of MFs and the topological
superconductor, which distinguishes it from all previous works. What we
need is to measure the electron occupation on QDs with charge sensing
techniques. Charge sensing has been developed intensively so far since
it is crucial for superconducting charge qubit {[}24{]} and quantum dot
qubit {[}25{]}. In particular, charge sensing based on the single
electron transistor (SET) has been well established {[}26{]}. It is a
circuit with a metallic island, and its conductance is sensitive to
nearby charges. When put close to a QD, the conductance measured from
the SET will be different for different electron occupations on the QD.
With a well designed integrated radio-frequency SET, it has been
demonstrated that real-time electron detection on QD can be achieved
with very low charge noise of
\(\delta q= 10^{-5} e {\rm Hz}^{\rm -1/2}\) within microsecond {[}27{]}.
With this technique, one can measure the electron occupation of the two
QDs in our setup simultaneously, and evaluate the conditional occupancy
probability \(P_1\) and \(P_2\) by repeating the measurements, which
determines the QD entanglement.''

\begin{enumerate}
\def\labelenumi{(\arabic{enumi})}
\setcounter{enumi}{4}
\item
  In the list of references, we add J. Linder, Y. Tanaka, T. Yokoyama,
  A. Sudb\{\o\}, and N. Nagaosa, Phys. Rev.~Lett. \{\bf 104\}, 067001
  (2010). (Ref.\textasciitilde{}17 in the revised manuscript).
  \vspace{5mm}
\item
  In the list of references, we add Y. Makhlin, G. Sch"\{o\}n, and A.
  Shnirman, Rev.~Mod. Phys. \{\bf 73\}, 357 (2001).
  (Ref.\textasciitilde{}24 in the revised manuscript).
\item
  In the list of references, we add A. A. Clerk, M. H. Devoret, S. M.
  Girvin, F. Marquardt, and R. J. Schoelkopf, Rev.~Mod. Phys.
  \{\bf 82\}, 1155 (2010). (Ref.\textasciitilde{}26 in the revised
  manuscript).
\end{enumerate}

\hypertarget{prl-round-2}{%
\subsection{PRL Round 2}\label{prl-round-2}}

\begin{longtable}[]{@{}l@{}}
\toprule
\endhead
\#\#\# Second Report of Referee A -- LL13514/Wang\tabularnewline
\bottomrule
\end{longtable}

This is report to resubmitted manuscript. Since resubmitted version is
not much different from the original one, I list below my comments to
the original version and access the authors replies to these comments (I
note that in the author's reply indexing of my comments is somewhat
shifted if not chaotic, so I use my original indexing).

\begin{enumerate}
\def\labelenumi{\arabic{enumi}.}
\tightlist
\item
  ``The Hamiltonian (5) does not conserve number of electrons\ldots{}.''
\end{enumerate}

This comment was incorrect, as the authors explained in their reply. I
agree with them on this point

\begin{enumerate}
\def\labelenumi{\arabic{enumi}.}
\setcounter{enumi}{1}
\tightlist
\item
  ``It is quite unclear in which respect the 4x4 Hamiltonian (8) is
  related to the presence of Majorana states\ldots{}.''
\end{enumerate}

I do not consider authors reply convincing on this point.

Reduction to this Hamiltonian is indeed possible under the condition
that some low-energy (much below superconductive gap) electron states is
allowed in the considered system. However, existence of such a states,
being compatible with Majorana nature of such states, still does not
prove it in any sense.

\begin{enumerate}
\def\labelenumi{\arabic{enumi}.}
\setcounter{enumi}{2}
\tightlist
\item
  ``The paper does not contain realistic estimates of the
  scales\ldots{}.''
\end{enumerate}

In the updated version the authors include statement ``The energy gap of
the topological superconductor has been estimated {[}7{]} to be around
15 K'' and then present some proposals of the temperature range.

To begin with, paper {[}7{]} DOES NOT contain the statement the present
authors would like to see there. 15 K mentioned in Ref.{[}7{]} is just
the energy gap of Nb, and then there is some speculation that energy gap
induced in InAs wire in contact with Nb might be of similar magnitude.

Thus the cited reference to {[}7{]} is too far-stretching, at least.
Even if induced gap in InAs can be made about 10 K, there is still
another issue: in order to produce ``topological superconductor'' from
this combination of bulk Nb and InAs wire, one should apply magnetic
field and choose its magnitude in the appropriate range. As a result,
the energy gap of the ``topological superconductor'', if it exists, will
be considerably smaller than 10 K.

\begin{enumerate}
\def\labelenumi{\arabic{enumi}.}
\setcounter{enumi}{3}
\tightlist
\item
  ``It is quite unclear which specific type of measurement\ldots{}.''
\end{enumerate}

The authors added some more discussion at the corresponding point. Still
all this discussion refers to measurement of different charge states of
the device (SET technique). What is really unclear in their proposal is
the relation between P\_1 - P\_2 and quantum entanglement which they
wish to study. Once again: correlations between occupation numbers of
electrons in the two QD do not measure entanglement, these correlation
reflect simple Coulomb blockade phenomenon which has nothing to do, in
my opinion, with exotic Majorana features discussed.

To conclude: the paper is not suitable for publication in the present
form. The major concern is about an issue 4 which should be clarified,
since without such a clarification there is no message whatsoever in the
paper. If this issue will be clarified, the paper may be published in
Physical Review as a regular one.

\begin{longtable}[]{@{}l@{}}
\toprule
\endhead
\#\#\# Second Report of Referee B -- LL13514/Wang\tabularnewline
\bottomrule
\end{longtable}

I acknowledge the authors for their clarifications, and I understand
that they propose the measurement of a double dot system occupancies
instead of transport. I believe this is an interesting idea although
actual feasibility might depends a lot on the experimental efficiency in
realizing such kind of detection (I mean I am not sure the proposed
experience would be simpler to realize than transport or spectroscopic
measurements).

I think the manuscript is now sufficiently improved to deserve
publication in Phys. Rev B, but to my opinion, the main theoretical idea
still lacks enough novelty (compared to the respective works of Fu and
Flensberg) in order to justify publication in Phys. Rev.~Letter.

    \hypertarget{ratchet-potential-and-rectification-effect-in-majorana-fermion-squid}{%
\section{Ratchet potential and rectification effect in Majorana fermion
SQUID}\label{ratchet-potential-and-rectification-effect-in-majorana-fermion-squid}}

\hypertarget{prl-round-1}{%
\subsection{PRL Round 1}\label{prl-round-1}}

\begin{longtable}[]{@{}l@{}}
\toprule
\endhead
\#\#\# Report of Referee A -- LR13604/Wang\tabularnewline
\bottomrule
\end{longtable}

I have read carefully the article of Z. Wang et al.~and I have concluded
that it is not ready for publication. Quite frankly I fear the authors
have ignored an important part of their system, which I suspect may
completely alter their results. Consequently, I am very skeptical what
the authors predict will actually be seen in experiment. The paper needs
a lot more work before it will be ready for publication, and I suspect
the results will be significantly different.

My first and most serious comment is that the authors have completely
neglected what happens at the edges of the semiconductor nanowire. The
Kitaev model does not smoothly deform into an ordinary s-wave
superconductor, and in fact the two should be separated by a domain wall
with gapless localized quasiparticle states. This must be the case since
the Kitaev wire is a topological state, while the s-wave superconductor
is not. Given these additional low energy states, the authors need to
repeat their analysis taking these other low energy states into account.
The results may be completely different as a consequence.

A number of articles in the recent literature concerning Majorana
fermion physics were not cited, and I think should be in this article:

\begin{enumerate}
\def\labelenumi{\arabic{enumi})}
\item
  J. Alicea, Phys. Rev.~B 81, 125318 (2010)
\item
  Jiang et al.~Phys. Rev.~Lett. 107, 236401 (2011)
\end{enumerate}

Eq. (11) is clearly not simplified to the full extent that it could be,
obscuring the analysis. Sadly, this also suggests the authors have not
been as careful as they should be in the analysis.

Finally how special is the \delta\_1 = \delta\emph{2 case? Could the
authors repeat their simple analysis with \delta}\{1,2\} not equal?

\hypertarget{remarks-intended-solely-for-the-editor}{%
\subsubsection{Remarks intended solely for the
editor:}\label{remarks-intended-solely-for-the-editor}}

Dear Dr.~Donavan Hall Senior Assistant Editor Physical Review Letters

Thank you for your message as of May 14th, on our manuscript LR13604
entitled as ``Ratchet potential and rectification effect in Majorana
fermion SQUID'' by Z. Wang, Q.-F. Liang and X. Hu.

We have read the Referee comments carefully, and found that his/her
concerns can be resolved fully, and our results remain unchanged. It
seemed to him/her that the Majorana fermions at the two ends of the
nanowire had been ignored in our work, and s/he thought that including
these two end MFs may influence the result significantly. It is actually
not the case. Namely, the 4x4 Hamiltonian including two end MFs and the
two MFs at the junctions is block diagonal for the appropriate basis,
with both the upper and lower 2x2 matrices equivalent to the two-MF
model presented in our submitted manuscript. Following the suggestions
of the Referee, we extend our formalism by explicitly including totally
four MFs, which improves the presentation of our results. For details
please refer to the report to Referee and the summary of changes.

We believe that all the comments from the Referee have been addressed,
and hereby resubmit our manuscript for your further consideration.

With warmest regards,

Sincerely yours,

Xiao Hu on behalf of all authors

\hypertarget{reply-to-referee-a}{%
\subsubsection{Reply to Referee A}\label{reply-to-referee-a}}

\begin{quote}
Comment1: I have read carefully the article of Z. Wang et al.~and I have
concluded that it is not ready for publication. Quite frankly I fear the
authors have ignored an important part of their system, which I suspect
may completely alter their results. Consequently, I am very skeptical
what the authors predict will actually be seen in experiment. The paper
needs a lot more work before it will be ready for publication, and I
suspect the results will be significantly different. My first and most
serious comment is that the authors have completely neglected what
happens at the edges of the semiconductor nanowire. The Kitaev model
does not smoothly deform into an ordinary s-wave superconductor, and in
fact the two should be separated by a domain wall with gapless localized
quasiparticle states. This must be the case since the Kitaev wire is a
topological state, while the s-wave superconductor is not. Given these
additional low energy states, the authors need to repeat their analysis
taking these other low energy states into account. The results may be
completely different as a consequence.
\end{quote}

Reply1: First, we thank the Referee for the careful reading. We
understand that there are domain walls at the edges of the nanowire, and
end Majorana fermions exist there. They correlate with the Majorana
fermions at the junction area and in principle may influence the
current-phase relation of the Majorana Josephson junction. For the
Majorana Josephson junction, we adopted a minimal model, in which these
influences were included through a phenomenological perturbation term.
We wish to say that this minimal model is sufficient to grasp the
physics of the Majorana Josephson junction. We should also admit that
this is not a trivial case, as pointed out by this Referee. In order to
make this point clear, we include the two end Majorana fermions
explicitly. The lowest order perturbation theory then is described by a
4x4 matrix. It is shown explicitly that the 4x4 matrix is block diagonal
when the basis is chosen for the electronic occupation, with the upper
and lower 2x2 matrices referring to the even and odd parity,
respectively. Since the total Hamiltonian commutes with the parity
operator, one can focus on each of the parities separately. Therefore,
the system of four Majorana fermions is reduced to the 2x2 matrix, which
is equivalent to our previous simplified formalism. As for the SQUID, we
consider that the maximum supercurrent at domain walls between the
conventional superconductor and the topological superconductor is much
larger than the critical current of the Majorana Josephson junction. For
the detail, please refer to the revised manuscript.

Comment2: A number of articles in the recent literature concerning
Majorana fermion physics were not cited, and I think should be in this
article: 1) J. Alicea, Phys. Rev.~B 81, 125318 (2010) 2) Jiang et
al.~Phys. Rev.~Lett. 107, 236401 (2011)

Reply2: We thank the Referee for this suggestion, and we apologize that
we had not cited these two works. We include these papers in our
reference. For detail, please refer to the revised manuscript and
summary of changes.

Comment3: Eq. (11) is clearly not simplified to the full extent that it
could be, obscuring the analysis. Sadly, this also suggests the authors
have not been as careful as they should be in the analysis.

Reply3: We would like to thank the Referee for his/her careful reading.
A simplification according to this suggestion makes physics transparent.
For details, please refer to the revised manuscript and the summary of
changes.

Comment4: Finally how special is the \(\delta_1 = \delta_2\) case? Could
the authors repeat their simple analysis with \(\delta_{1,2}\) not
equal?

Reply4: Yes, we can. Generally, the phase \(\phi\) of the off-diagonal
term in the 2x2 matrix can be arbitrary, which can be absorbed through a
gauge transformation
\(P_1\rightarrow P_1, P_2\rightarrow e^{i\phi} P_2\), which does not
change the Josephson current given by \(|P_1|^2\) and \(|P_2|^2\) (see
Eq.(7)). The gap opened by the off-diagonal terms is proportional to
their norm, which determines the physics of the present system.

At last, we thank the Referee for the careful reading and helpful
comments. We believe that we have addressed all the concerns of the
Referee, and appreciate very much for his/her reconsideration.

Summary of changes:

\begin{enumerate}
\def\labelenumi{(\arabic{enumi})}
\item
  In abstract, we change `Josephson current' to `current-phase relation'
\item
  Fig. 1 has been changed. Meanwhile, in its caption, we change
  `containing two Majorana fermions' to `containing Majorana fermions'
\item
  The second section ``current-phase relation'' is reformulated. We
  extend the model and include the two end Majorana fermions explicitly,
  and describe the system by a 4x4 matrix. The Eqs. (1), (3), (6), (7),
  (8), (10) in our revised manuscript have been modified, with their
  descriptions and explanations changed accordingly.
\item
  In the first paragraph of the `Asymmetric critical current' section,
  we add `We consider that the maximum supercurrent at domain walls
  between the conventional superconductor and the topological
  superconductor is much larger than the critical current of the
  Majorana Josephson junction.'
\item
  In the first paragraph of the `Rectification effect' section, we
  change `where the translational anti-symmetry
  \(I(\theta+\pi)=-I(\theta)\) enjoyed by conventional SQUIDs is
  broken.' to `where the translational anti-symmetry
  \(I(\theta+\pi)=-I(\theta)\) and the reversal anti-symmetry
  \(I(-\theta)=-I(\theta)\) enjoyed by conventional SQUIDs are broken.'
\item
  In the caption of Fig. 2-4, we change `\(\delta=0.01T\)' to
  `\(\delta=0.014T\)'
\item
  We change the label of the vertical axis in Fig. 3 and Fig. 4.
\item
  In the list of references, we add J. Alicea, Phys. Rev.~B \{\bf 81\},
  125318 (2010). (Ref.\textasciitilde{}7 in the revised manuscript).
\item
  In the list of references, we add L. Jiang, D. Pekker, J. Alicea, G.
  Refael, Y. Oreg, and F. von Oppen, Phys. Rev.~Lett. \{\bf 107\},
  236401 (2011). (Ref.\textasciitilde{}12 in the revised manuscript).
\item
  In the list of references, we add P. San-Jose, E. Prada, and R.
  Aguado, arXiv:1112.5983. (Ref.\textasciitilde{}27 in the revised
  manuscript).
\item
  There are also some minor grammatical changes.
\end{enumerate}

\hypertarget{prl-round-2}{%
\subsection{PRL Round 2}\label{prl-round-2}}

\begin{longtable}[]{@{}l@{}}
\toprule
\endhead
Report of Referee B -- LR13604/Wang\tabularnewline
\bottomrule
\end{longtable}

In the submitted paper the authors have tried to detect the presence of
Majorana bound states via the asymmetric current phase relation. The
basic premise of the paper is wrong.

An asymmetric current phase relation can be observed in many Josephson
junctions for example with d-wave superconductors. In s-wave for the
case when magnitude of superconducting gap on either side of insulator
is unequal then too an asymmetric CPR can be seen. The RMP by Golubov
et. al in 2004 has this in much more detail.

Further, this case of delta\_1 \neq \delta\_2 has been raised by the
first referee and unfortunately the authors haven't answered properly.
The Josephson current will change.

The authors have also not explained the steps in calculation from eq.
6-eq.8. Perhaps they should put it in supplementary information. In the
last line of page 2 they mention that Josephson current is given by
diagonalizing eq 8. This is not correct. Since eq 8 is the Schrodinger
eqn. Diagonalizing 8 will give Andreev bound states.

Finally the rectification effect predicted because of ac Josephson
effect in analogy to another work lacks substance as a detection tool.
Since it is based on false premise of being unique only to Majorana
bound state in Josephson junction squid. As mentioned earlier asymmetry
in CPR is seen in wide range of cases. A unique method of detecting
Majorana bound states has been shown in PRB 81, 085101 (2010). The
authors could perhaps look into that work and check whether something
uniquely due to Majorana bound states can also be seen in their set-up.

I have to reject the paper as such but if the authors modify their paper
and give a unique criteria for their detection and also expand upon the
theoretical derivation of their Eqns 8-10, then I would be prepared to
reconsider.

\hypertarget{details-of-changes}{%
\subsection{Details of changes:}\label{details-of-changes}}

\hypertarget{report-of-referee-a-lr13604wang}{%
\subsection{Report of Referee A --
LR13604/Wang}\label{report-of-referee-a-lr13604wang}}

Comment A1: I have read carefully the resubmitted article of Z. Wang et
al.~I am happy to say that a lot of problems in the first version have
been dealt with. However, the article remains difficult to read and many
of the key points are not explained as well as they could be. Indeed, it
took me some time to really understand what the authors have done. I
think the paper still needs to improve before publication.

Reply A1: First, we thank this Referee for the careful reading, and
acceptance of our responses to his/her previous comments. Following
his/her suggestions, we revise the manuscript to make a clear
explanation on our main results. For details please see the following
replies, and the summary of changes and the manuscript.

Comment A2: The key physical discovery is that an asymmetric critical
current will appear in the squid device they consider, and that if found
in an experiment, it would be evidence for Majorana fermions in the
system. This asymmetry could be far more clearly introduced, and the
paper better motivated, by first saying that if a squid device with one
branch having the normal Josephson effect and the other having the
4\pi Josephson effect, one would get this asymmetry. Instead, we get
mired in details right at the beginning, and this point is obscured.

Reply A2: We appreciate very much that this Referee catches the key
points of our present work, and his/her clear indication on the way to
improve the presentation. We simplify the introduction part of the
manuscript, emphasizing on the key points of the present work and
shrinking detailed discussions. For details please see the summary of
changes and the manuscript.

Comment A3: This together with the discussion about adiabatic processes
vs.~fast processes makes the article hard to read, and unfortunately
also makes it readable only to specialists. I think this article could
be significantly improved by revamping the discussion and I sincerely
hope the authors can do so.

Reply A3: We thank the Referee for this suggestion. We modified the
discussion, making the key point clear and accessible to general
audiences. For details please see the summary of changes and the
manuscript.

Once again we thank the Referee for careful reading, capture of our main
results, and helpful suggestions for presentation. We hope that the
revised manuscript accommodates well the suggestions, and the
presentation is now much better and thus appeals to broad readers. We
would appreciate very much for his/her further consideration and
recommendation for publication.

\begin{longtable}[]{@{}l@{}}
\toprule
\endhead
Report of Referee B -- LR13604/Wang\tabularnewline
\bottomrule
\end{longtable}

Comment B1: In the submitted paper the authors have tried to detect the
presence of Majorana bound states via the asymmetric current phase
relation. The basic premise of the paper is wrong. An asymmetric current
phase relation can be observed in many Josephson junctions for example
with d-wave superconductors. In s-wave for the case when magnitude of
superconducting gap on either side of insulator is unequal then too an
asymmetric CPR can be seen. The RMP by Golubov et. al in 2004 has this
in much more detail.

Reply B1: We thank this Referee for raising questions on our work.
However, we cannot agree with the comment that the basic premise of the
paper is wrong.

First, please let us make it clear that there is no asymmetric current
phase relation in a single junction involving Majorana fermions. It only
exhibits an unconventional CPR with 4-pi period. The key point of our
present work is that when a MF junction is combined with a conventional
SIS Josephson junction into to SQUID, an asymmetric critical current
with respect to the current direction appears.

We propose to use this property to probe the Majorana fermions, since
successful synthesis of Josephson junction between two topological
superconductors has been reported recently (see the paper by Kouwenhoven
group, Science vol.~336 (2012) 1003), and only one further step is to be
taken, namely combining it with a conventional Josephson junction into a
SQUID, to provide another signature for Majorana fermions beside the
zero-bias conductance, the only evidence for existence of Majorana
fermions so far. Therefore, our proposal is experimentally relevant, and
timely.

Although an unconventional current-phase relation (of 4-pi period) can
happen in some other cases, such as d-wave superconductivity, and
ballistic limit of point-contact junction which have been discussed in
RMP by Golubov et. al in 2004, they can be distinguished clearly from
the present case caused by Majorana fermions, following the way used by
Kouwenhoven group (Science vol.~336 (2012) 1003). Namely, by tuning
either the magnetic field or the chemical potential on the nanowire, one
can switch the system between a topological state carrying Majorana
fermions, which shows current-phase relation of 4pi period, and a
trivial state without Majorana fermions, which shows conventional
current-phase relation as reported by J. A. V. Dam et al (Nature
vol.~442 (2006) 667).

In this sense, our proposal can provide unique, phase-sensitive evidence
for Majorana fermions. In order to make this point clear, we include a
discussion in our manuscript. For details please see the summary of
changes and the manuscript.

We also wish to point out that in an SIS Josephson junction, the CPR
should be conventional, i.e.~of 2 pi period, and the critical current
should be symmetric with respect to the current direction, even when the
s-wave superconducting gaps on the two sides of insulator are unequal.

Comment B2: Further, this case of delta\_1 \neq \delta\_2 has been
raised by the first referee and unfortunately the authors haven't
answered properly. The Josephson current will change.

Reply B2: As can be seen in Eq. (7), the summation and the difference of
the two interactions delta\_L and delta\_R appear at the off-diagonal
terms of the matrix. When one of them becomes zero accidentally, the
off-diagonal terms in one of the two blocks disappear, which then falls
into the fast-dynamics limit addressed in our manuscript. Otherwise, one
may have the adiabatic limit depending on the values of the off-diagonal
terms. In any case, the main point of our work, namely existence of the
direction-asymmetric critical current remains unchanged as revealed
clearly in the manuscript.

In the simplified model of two Majorana fermions described in the first
version of our manuscript, delta\_1 and delta\_2 correspond to the real
and imaginary parts of the off-diagonal terms respectively. In this
case, the unequal interactions raise a phase unequal to pi/4. However,
this phase is absorbed by a gauge transformation, and leaves no
consequence in the time evolution of Majorana state and the Josephson
current.

Comment B3: The authors have also not explained the steps in calculation
from eq. 6-eq.8. Perhaps they should put it in supplementary
information. In the last line of page 2 they mention that Josephson
current is given by diagonalizing eq 8. This is not correct. Since eq 8
is the Schrodinger eqn. Diagonalizing 8 will give Andreev bound states.

Reply B3: Diagonalizing Eq.(7) (8 in previous version) one obtains the
time evolution of the MF ground states, which then gives the Josephson
current by the formula below Eq.(7).

In conventional s-wave superconductors, Josephson current can be
understood as carried by finite-energy Andreev bound states derived in
the Bogoliubov-de Gennes formalism with respect to the whole Hamiltonian
including the bulk ones and the tunneling one (see for example A.
Furusaki and M. Tsukada, Physica B 165-166, 967 (1990)) at non-vanishing
phase difference. The result is equivalent to those based on
perturbation analyses for tunneling junctions, like the one given by
Ambegaoka and Baratoff (V. Ambegaoka and A. Baratoff, Phys. Rev.~Lett.
10, 486 (1963)). The Josephson current is proportional to T\^{}2/\Delta,
which reflects a second-order process corresponding to Cooper-pair
tunneling.

The situation is different in the case of Josephson junction involving
Majorana fermions, which form the degenerate ground state and make
dominant contributions to Josephson current. This Majorana-fermion
assisted tunneling process is revealed in our manuscript around Eqs.
(6)-(10) in terms of the perturbation approach with respect to the
tunneling Hamiltonian, which is similar to the one used by Ambegaoka and
Baratoff. In contrast to conventional junctions, however, the Josephson
current carried by the Majorana fermions is linear to the tunneling
matrix T, which reflects a first-order process corresponding to
single-electron tunneling, due to the degeneracy in the ground state.
For the fast-dynamics limit, a same conclusion has been obtained by
Alicea et al (Nat. Physics, vol.~7 (2011) 412) from the calculation of
energy spectrum of Majorana fermions.

We revise our manuscript and make clearer the unique property of
junctions with Majorana fermions. For details please see the summary of
changes and the manuscript. Following the suggestion of this Referee, we
prepare a note for the derivations between Eq.6-Eq.8 based on the
time-dependent perturbation approach. However, we have to say that the
derivations are standard, except that attention should be paid to the
zero-energy degeneracy, as described clearly in the main text.

Comment B4: Finally the rectification effect predicted because of ac
Josephson effect in analogy to another work lacks substance as a
detection tool. Since it is based on false premise of being unique only
to Majorana bound state in Josephson junction squid. As mentioned
earlier asymmetry in CPR is seen in wide range of cases.

Reply B4: As discussed in Reply B1, possibilities like those involving
d-wave superconductivity and ballistic limit of SNS junction, can be
excluded by tuning either the magnetic field or the chemical potential
on the nanowire, which switches the system between topological and
trivial states. In this sense, one can obtain unique, phase sensitive
evidence for the Majorana fermions.

Comment B5: A unique method of detecting Majorana bound states has been
shown in PRB 81, 085101 (2010). The authors could perhaps look into that
work and check whether something uniquely due to Majorana bound states
can also be seen in their set-up.

Reply B5: We thank this Referee for bringing our attention to this work
which is now included in our reference list. The proposal in PRB 81,
085101 (2010) is based on topological insulator / superconductor
heterostructure. It seems that realization of this structure with
Majorana bound states is still challenging experimentally. In contrast,
our proposal is based on the structure already achieved experimentally
by the group headed by Kouwenhoven in 2012. They successfully
synthesized the topological superconductor (TS) nanowire / normal metal
junction, and Josephson junction between two topological superconductors
as well. The zero-bias conductance peak at the TS-metal junction has
been observed, which is an important signal for Majorana fermions. In
the present work, we propose to combine the experimentally available TS
Josephson junctions with a conventional Josephson junction, and to
measure the critical currents in opposite current directions. Therefore,
our proposal is experimentally relevant, and timely.

Comment B6: I have to reject the paper as such but if the authors modify
their paper and give a unique criteria for their detection and also
expand upon the theoretical derivation of their Eqns 8-10, then I would
be prepared to reconsider.

Reply B6: As discussed above, our proposal, with additional explanations
following the suggestions from this Referee, can give unique criteria
for detection on MFs. The derivation of Eq. 6 8 is also enhanced. We
hope that these revisions can accommodate the comments from this
Referee, and our work meets the standard of PRL. His/her reconsideration
is much appreciated.

\begin{longtable}[]{@{}l@{}}
\toprule
\endhead
Summary of Changes -- LR13604/Wang\tabularnewline
\bottomrule
\end{longtable}

1)The title is slightly changed in order to convey more clearly our main
point. 2)The abstract is rephrased, concentrating on the asymmetric
critical current, which is the key result of our work. 3)In the
Introduction section, the third paragraph is reformed, emphasizing on
the key idea, and shrinking the discussion on details. 4) In the caption
of Fig. 1, we change, Schematic setup of SQUID composed of a novel
junction containing Majorana fermions and a conventional SIS junction.
to Schematic setup of SQUID with a novel junction containing Majorana
fermions in one branch and a conventional SIS junction in the other
branch. 5)In the Current-phase relation section, the Eqs. (5)-(8) of the
old manuscript are rearranged, and the discussion around these equations
are reformulated, in order to clarify our approximation in the
calculation of Josephson current. 6) In the Current-phase relation
section, Just before Eq. (10) of the revised manuscript, we change,
However, for the slow adiabatic process, the instant ground state of the
2x2 matrix is always reached, and the Josephson current is given by
diagonalizing Eq.\textasciitilde{}(8), to For adiabatic process, the
instant ground state of the 2x2 matrix is always reached, which can be
given by diagonalizing Eq.\textasciitilde{}(7). Plugging the
ground-state wave function thus obtained into the current expression, we
arrive at 7)In the Current-phase relation section, directly after Eq.
(10) of the revised manuscript, we change, It is clear that the CPR of
the MF Josephson junction strongly depends on the dynamics of the end
MFs, and generally deviates from the simple sinusoidal function of phase
difference realized in conventional Josephson junctions. to For
high-quality MF junctions, the interactions from the two far ends are
very small, where the \(4\pi\) period CPR should be observed. 8)In the
Asymmetric critical current section, the beginning of the penultimate
paragraph, we change, both originated from the existence of MFs as
discussed above. to both originated from MFs as discussed above, which
breaks in Eq.\textasciitilde{}(10) the translational anti-symmetry
\(I(\theta+\pi)=-I(\theta)\) and the reversal anti-symmetry
\(I(-\theta)=-I(\theta)\) enjoyed by conventional SQUIDs
\cite{quintero}. 9)In the Asymmetric critical current section, the end
of the penultimate paragraph, we change, Observable both for fast and
adiabatic dynamics, the asymmetry of critical Josephson current with
respect to the flowing direction is ubiquitous for MF systems, as
compared with the fractional Josephson effect which is expected only for
fast dynamics. to Observable both for fast and adiabatic dynamics, the
asymmetry of critical Josephson current with respect to the flowing
direction is ubiquitous for MF systems. 10)In the Asymmetric critical
current section, the end of last paragraph, we add, Moreover, it has
been shown that the MF Josephson junction can be tuned between
topological trivial and non-trivial states. Therefore, this asymmetric
critical current, which only appears at the topological non-trivial
state, provides a unique evidence for MFs in the system. 11) In the
Rectification effect section, the first paragraph, we delete, due to the
unconventional CPRs in Eqs.\textasciitilde{}(9) and/or (10), where the
translational anti-symmetry \(I(\theta+\pi)=-I(\theta)\) and the
reversal anti-symmetry \(I(-\theta)=-I(\theta)\) enjoyed by conventional
SQUIDs are broken\cite{quintero}. 12)We reformed the Conclusion section,
highlighting more clearly our main contributions. 13)In the list of
references, we add C. Benjamin and J. K. Pachos, Phys. Rev.~B
\textbf{81}, 085101 (2010). (Ref.\textasciitilde{}8 in the revised
manuscript). 14)In the list of references, we add A. A. Golubov, M. Yu.
Kupriyanov, and E. Ilichev, Rev.~Mod. Phys. \textbf{76}, 411 (2004).
(Ref.\textasciitilde{}27 in the revised manuscript). 15) There are also
some minor grammatical changes.

Remarks intended solely for the editor: Dear Dr.~Donavan Hall Senior
Assistant Editor Physical Review Letters

Thank you for your email contacts as of July 10 and 13 on our manuscript
Ratchet potential and rectification effect in Majorana fermion SQUID, by
Zhi Wang, Qi-Feng Liang, and myself with the code LR13604.

We find that in the second report Referee A evaluates our work very
positively. He points out that the direction asymmetry in critical
current of MF SQUID, when found in an experiment, is an important
evidence for Majorana fermion. He also indicates helpfully the way to
improve the manuscript in order to highlight this key physical
discovery.

We read the report from Referee B carefully. Unfortunately, we find this
report not solid at least. His/her criticism on the basic premise of the
present work by raising the possibility of d-wave superconductivity is
irrelevant, since it hardly happens in the present system. His/her
statement on the existence of asymmetric critical current in an SIS
Josephson junction with different superconducting gaps on the two sides
of insulator is incorrect.

On the other hand, Referee B requests us to discuss whether we can make
our proposal a unique evidence for Majorana fermion, which is actually a
very constructive suggestion. This protocol is available as already
adopted by Kouwenhoven et al. (Science vol.~336 (2012) 1003), namely
switching the system from topological state to trivial state will
diminish the Majorana fermions and thus all the possible signals from
them, which makes the signals unique for Majorana fermions. Including
the discussion on this protocol into manuscript enhances our
presentation.

We have replied all the comments from the Referees, and revised our
manuscript according to the very positive indications of Referee A, as
well as the constructive one from Referee B. In order to convey more
clearly our main point, we modify the title of our manuscript slightly.

As you know, the importance of Majorana modes in topological quantum
computation is appreciated very much recently, as signaled by Kitaevs
winning of fundamental physics prize. Detecting Majorana fermion becomes
one of the topmost issues in condensed matter physics, and the race is
accelerated by the work by the group of Kouwenhoven. Our proposal on a
SQUID involving Majorana fermions is a unique, phase-sensitive protocol,
and would be interesting for both theoretical and experimental
physicists, as well as those working for materials science and nano
technology, and thus meets the standards of PRL.

Your further consideration is very much appreciated.

Sincerely yours, Xiao Hu On behalf of all authors

\hypertarget{prl-round-3}{%
\subsection{PRL Round 3}\label{prl-round-3}}

\begin{longtable}[]{@{}l@{}}
\toprule
\endhead
\#\#\# Report of Referee C -- LR13604/Wang\tabularnewline
\bottomrule
\end{longtable}

In their manuscript Wang, Liang and Hu (hereafter `the authors')
consider, theoretically, a dc-SQUID which is composed of conventional
Josephson junction (JJ) on one side and a Majorana nanowire on the
other. If compared to a regular dc-SQUID, the novel feature of this
setup is the inclusion of the wire which serves as a 1D Majorana JJ.

I personally find that the authors' claim that such setup can be used to
detect Majoranas (e.g. ``this phenomenon would provide evidence for
Majorana fermions'' from the abstract and ``If found\ldots{} it is
evidence for Majorana fermions\ldots{}'' from the conclusion) is, to say
the least, unjustified. In particular, asymmetry of the critical current
- the claimed manifestation of Majoranas - has been known in the
superconducting literature since late 1960s. I also believe that some of
authors' statements actually contradict their findings. Given the
abundance of proposals to detect 1D Majoranas and, at the very best,
incremental role of this one, I cannot recommend its publication in PRL.

Let me elaborate.

\begin{enumerate}
\def\labelenumi{\arabic{enumi})}
\item
  The field asymmetry of the critical current in JJs and SQUIDs is not
  at all unique to Majoranas, see, e.g.~Yamashita and Onodera, J. Appl.
  Phys. 38, 3523 (1967) and also the book of Kulik and Yanson, Josephson
  Effect in Superconducting Tunneling Structures, 1972. This implies
  that such feature cannot be used as a decisive signature, contrary to
  the authors' believe. Also, as has been pointed out by Referee B, the
  asymmetry of CPRs is well known as well and does not require Majorana
  JJs. So, the optimism of the authors that asymmetries ``can serve as
  and ideal probe for the MFs'' (yet another quote) is very much
  unfounded.
\item
  It seems to me that the authors' statement that ``the interplay
  between \ldots{} 4pi period and the well-known 2pi period \ldots{}
  leads to the asymmetry\ldots{}'' actually contradicts their findings.
  In fig 1a, the CPR of the adiabatic process is 2pi periodic. Yet, in
  fig 2d Ic is obviously asymmetric!
\end{enumerate}

Moreover, the saw-tooth CPR computed by the authors for an adiabatic
Majorana JJ is well-known for long superconducting wires (and other
systems as well). Given that the authors actually consider a wire with
the proximity-induced superconductivity, seeing such a CPR cannot, at
any rate, be considered as a direct signature of Majoranas.

\begin{enumerate}
\def\labelenumi{\arabic{enumi})}
\setcounter{enumi}{2}
\tightlist
\item
  Finally, let me point out that a much simpler experiment that can do
  the trick. CPRs of various junctions have been routinely measured for
  many years using rf-SQIUDs - a ring with only one JJ - see,
  e.g.~Frolov et al.~Phys. Rev.~B 70, 144505 (2004). This type of
  experiment can directly demonstrate the 4pi periodicity and, hence, is
  a much more plausible tool for experimental detection of Majorana JJs.
\end{enumerate}

\begin{longtable}[]{@{}l@{}}
\toprule
\endhead
\#\#\# Reply of Referee C -- LR13604/Wang\tabularnewline
\bottomrule
\end{longtable}

Comment C1: I personally find that the authors' claim that such setup
can be used to detect Majoranas (e.g. ``this phenomenon would provide
evidence for Majorana fermions'' from the abstract and ``If
found\ldots{} it is evidence for Majorana fermions\ldots{}'' from the
conclusion) is, to say the least, unjustified. In particular, asymmetry
of the critical current the claimed manifestation of Majoranas - has
been known in the superconducting literature since late 1960s. I also
believe that some of authors' statements actually contradict their
findings. Given the abundance of proposals to detect 1D Majoranas and,
at the very best, incremental role of this one, I cannot recommend its
publication in PRL.

Reply C1: We thank this Referee for raising questions on our
work.However, we cannot agree with the comment of this Referee. Although
asymmetric critical currents may appear in other systems due to various
mechanisms, such as magnetic inductance of strong current in a
conventional Josephson junction, or d-wave pairing, they can be
distinguished unambiguously from the present case which we are
discussing as an evidence of Majorana fermions (MFs). The key point here
is that the Josephson junction with spin-orbit-coupling nanowire
exhibits an unconventional current-phase relation (CPR) in its
topologically non-trivial state with MFs, and that the junction can be
easily tuned into a topologically trivial state with conventional CPR
(see J. A. V. Dam et al.~Nature vol.~442, 667, 2006) by simply rotating
the magnetic field applied to the nanowire (see the introduction of our
last manuscript). This switching between trivial and topological states
has been achieved by Kouwenhoven group (Science vol.~336 (2012) 1003)
experimentally, and served as the key ingredient of their method to
probe MFs through the zero-bias peak in conductance. Our setup shares
the same merit as this previous experiment, namely the asymmetric
critical current appears when the magnetic field is applied in parallel
to the nanowire, and will disappear when the magnetic field is vertical
to the nanowire. With this process, all other possible origins of
asymmetric critical current can be excluded, which leaves it as a clear
evidence for MFs.

There are a bunch of proposals for detecting MFs. Up to the time of this
writing, only the zero-bias peak in conductance found by Kowenhovens
group has been reported as a consistent signature for existence of MF.
The reason for this situation lies largely in the fact that many
proposals require severe conditions, which are not easy to achieve
experimentally. In the contrary, the SQUID setup proposed in our
manuscript does not require a severer condition than the experiment
realized by Kowenhoven et al.~We wish to notice that Kowenhoven et
al.~have reported the successful synthesis of Majorana Josephson
junction in their paper on Science. Here in the present work, we propose
to connect this Majorana Josephson junction with a conventional SIS
junction to form a SQUID, which is not a difficult task in experiment.
Then a measurement on the critical current, which is already a routine
procedure, will provide a phase-sensitive signature for MFs,
complementary to the zero-bias peak in conductance, the only evidence
for existence of MFs so far. Therefore, our proposal is experimentally
relevant, and timely.

Comment C2: The field asymmetry of the critical current in JJs and
SQUIDs is not at all unique to Majoranas, see, e.g.~Yamashita and
Onodera, J. Appl. Phys. 38, 3523 (1967) and also the book of Kulik and
Yanson, Josephson Effect in Superconducting Tunneling Structures, 1972.
This implies that such feature cannot be used as a decisive signature,
contrary to the authors' believe. Also, as has been pointed out by
Referee B, the asymmetry of CPRs is well known as well and does not
require Majorana JJs. So, the optimism of the authors that asymmetries
``can serve as and ideal probe for the MFs'' (yet another quote) is very
much unfounded.

Reply C2: We notice that the self-field effect of the Josephson junction
such as that reported by Yamashita and Onodera appears for large
super-current. This effect can be easily suppressed by setting small
critical super-current as in many junction experiments, where there is
no asymmetric critical current. In a sharp contrast, the asymmetric
critical current due to MFs persists to very small current, which can be
tuned straight forwardly by the gate voltage at the MF junction. As for
the comment from Referee B, we settled in the last Reply. In short, the
asymmetric critical current caused by the intrinsic property of the
junction with MFs can be distinguished easily from the one due to the
inductive field, and our proposal does serve as a signature for MFs.

Comment C3: It seems to me that the authors' statement that ``the
interplay between \ldots{} 4pi period and the well-known 2pi period
\ldots{} leads to the asymmetry\ldots{}'' actually contradicts their
findings. In fig 1a, the CPR of the adiabatic process is 2pi periodic.
Yet, in fig 2d Ic is obviously asymmetric!

Reply C3: There is no contradiction here. The key point of our work is
that interference between the unconventional CPR in MF Josephson
junction and the conventional CPR in an SIS junction will induce
asymmetric critical current, as summarized in the Abstract and
demonstrated clearly in the main body of our manuscript. The
unconventional CPR relation manifests itself most remarkably as a 4pi
period. In other cases, such as adiabatic dynamics which needs more
analyses (as in our manuscript), a complex wave-shape of 2pi appears,
which also leads to an asymmetric critical current. In order to avoid
making the situation too complex, we presented our main result, namely
the asymmetric critical current in a SQUID caused by MFs, in a most
direct way in introduction and conclusion. It was also a suggestion from
Referee A, which should make this work appeal to broader audiences. In
order to accommodate the comment of this Referee, we revise the
description on this point in conclusion. For those who are interested in
details, the main body of our manuscript gives the detailed discussion
(see sections Current-phase relation and Asymmetric critical current).
For details please see the summary of changes and manuscript.

Comment C4: Moreover, the saw-tooth CPR computed by the authors for an
adiabatic Majorana JJ is well-known for long superconducting wires (and
other systems as well). Given that the authors actually consider a wire
with the proximity-induced superconductivity, seeing such a CPR cannot,
at any rate, be considered as a direct signature of Majoranas.

Reply C4: As discussed in Reply C1, a key ingredient here is that the MF
Josephson junction can be tuned simply by rotating the magnetic field
along the nanowire between trivial state which has a conventional CPR,
and the topologically nontrivial state which exhibits an unconventional
CPR either of a 4pi period or a complex wave-shape with 2pi period. This
process can exclude other possible reasons causing the saw-tooth CPR.

Comment C5: Finally, let me point out that a much simpler experiment
that can do the trick. CPRs of various junctions have been routinely
measured for many years using rf-SQIUDs - a ring with only one JJ - see,
e.g.~Frolov et al.~Phys. Rev.~B 70, 144505 (2004). This type of
experiment can directly demonstrate the 4pi periodicity and, hence, is a
much more plausible tool for experimental detection of Majorana JJs.

Reply C5: As a matter of fact, detection of MFs based on rf-SQUID
measurement was proposed sometime ago (R. M. Lutchyn et al.~Phys.
Rev.~Lett. vol.105, 077001, 2010, Reference 16 in our manuscript).
However, up to this moment, no consistent signature has been reported
along this direction. This technique is therefore not as simple for MF
systems as it may sound, perhaps due to the fact that another circuit
has to be inductively coupled to the rf-SQUID including MFs (see Frolov
et al.~Phys. Rev.~B 70, 144505 (2004), the work raised by this Referee,
and Golubov et al.~Rev.~Mod. Phys 76, 411 (2004)). In contrary,
measuring the critical current for Josephson junction and SQUID as in
our proposal is a more routine experiment. Here we wish to note that it
has been performed for the topological Josephson junction by Kowenhovens
group (see the supplement of Science vol.~336 (2012) 1003). Generally
speaking, to map out the whole CPR requires a more sophisticated
technique than to detect critical currents. In this sense, our proposal
is considered to be more feasible experimentally.

\begin{longtable}[]{@{}l@{}}
\toprule
\begin{minipage}[b]{0.96\columnwidth}\raggedright
Summary of Changes -- LR13604/Wang\strut
\end{minipage}\tabularnewline
\midrule
\endhead
\begin{minipage}[t]{0.96\columnwidth}\raggedright
1)In the ninth line of the third paragraph of section Introduction, we
addin a high-quality MF junction after 4\(\pi\) period. 2)Under Eq. (6),
we add with \(J_M = e{T}/{\hbar}\). 3)Under Eq. (9), we add The CPR in
this case is also novel as compared with the conventional one
\(I_{\rm N}\sim \sin\theta\). 4)At the beginning of the third paragraph
of the section Asymmetric critical current, we add The physical origin
of this asymmetric critical current is the interplay between the
unconventional CPR of the MF junction and the well-known one
\(I_{\rm N}\sim \sin\theta_{\rm N}\) for SIS junction. 5)We delete in
this paragraph the sentence It is noticed that a SQUID formed by two
conventional Josephson junctions cannot realize this asymmetric CPR even
if they possess different critical currents. to save space. 6)In the
last part of the fourth paragraph of the section of Asymmetric critical
current, we revise the descriptions after Moreover, it has been shown
that . 7)We revise slightly the first part of the section of
Conclusion.\strut
\end{minipage}\tabularnewline
\begin{minipage}[t]{0.96\columnwidth}\raggedright
Remarks intended solely for the editor: Dear Dr.~Donavan Hall Senior
Assistant Editor Physical Review Letters\strut
\end{minipage}\tabularnewline
\begin{minipage}[t]{0.96\columnwidth}\raggedright
Thank you for your message as of Oct.~9th, on our manuscript
LR13604.\strut
\end{minipage}\tabularnewline
\begin{minipage}[t]{0.96\columnwidth}\raggedright
We have read the comments by Referee C, and found that his/her concerns
can be resolved fully. The effect s/he raised may appear in Josephson
junctions of large super-current, which is not the case in the context
of qubit and MF system. And it can be excluded unambiguously by a
protocol already mentioned in our manuscript.\strut
\end{minipage}\tabularnewline
\begin{minipage}[t]{0.96\columnwidth}\raggedright
His/her suggestion to use the rf-SQUID technique to detect MFs
unfortunately indicates that s/he is not familiar with MF systems, since
that technique was already proposed theoretically two years ago by R. M.
Lutchyn et al. (Phys. Rev.~Lett. vol.105, 077001, 2010, Reference 16 in
our manuscript). Up to this moment, there is no successful experimental
report on this proposal, which partially motivated us to propose the new
idea.\strut
\end{minipage}\tabularnewline
\begin{minipage}[t]{0.96\columnwidth}\raggedright
We respond to Referee C in details in our reply, and we believe that all
the comments have been addressed. Hereby we resubmit our manuscript to
PRL.\strut
\end{minipage}\tabularnewline
\begin{minipage}[t]{0.96\columnwidth}\raggedright
Thank you in advance for your kind consideration.\strut
\end{minipage}\tabularnewline
\begin{minipage}[t]{0.96\columnwidth}\raggedright
With warmest regards,\strut
\end{minipage}\tabularnewline
\begin{minipage}[t]{0.96\columnwidth}\raggedright
Sincerely yours,\strut
\end{minipage}\tabularnewline
\begin{minipage}[t]{0.96\columnwidth}\raggedright
Xiao Hu on behalf of all authors\strut
\end{minipage}\tabularnewline
\begin{minipage}[t]{0.96\columnwidth}\raggedright
\#\# PRL Round 4\strut
\end{minipage}\tabularnewline
\bottomrule
\end{longtable}

\hypertarget{second-report-of-referee-c-lr13604wang}{%
\subsection{\#\#\# Second Report of Referee C --
LR13604/Wang}\label{second-report-of-referee-c-lr13604wang}}

I have read the manuscript one more time; I have also looked over the
authors' reply to my comments.

Here is the punch line: I believe that conclusions of the manuscript are
flawed. Enclosed to this response is a counter example which
demonstrates this (in the pdf file).

Intentionally or not, in their manuscript and response, the authors use
sleight of hand. Typical example of this is their answer to my comment 3
from the previous round of reviews which was: How come both 2pi and 4pi
periodic CPRs show asymmetry in Ic if such asymmetry is (``an
unambiguous'') signature for MFs?

The authors start their answer with a definitive ``There is no
contradiction here'' and just a few lines below admit that ``\ldots{} a
complex wave-shape of 2pi appears which also leads to the asymmetry of
critical current'' (``complex'' here means complicated). I take this as
the coup de grace: This statement invalidates the main claim of the
manuscript since any practical (experimental) weak link, unlike
theoretical models, has a complex CPR which could be quite different
from a regular sine wave. Yes, MFs may produce the field asymmetry in
the critical current but so do billion other things! The authors rescue
boat - the tunability of MFs with the Zeeman coupling - is of no help
here.

The authors also want their work to look ``novel'', as if it appears in
vacuum, without mentioning previous conceptually very similar proposals
to detect MF. In fact, one of the such proposals has been realized (more
than half a year ago!) and which the authors ignore is reported here:
1204.4212.

Because of the factors mentioned above I feel quite strongly that the
manuscript should not be published in PRL. In fact, in its current form,
I would object to accepting it in any APS journal.

\hypertarget{reply-to-comments}{%
\subsubsection{Reply to comments:}\label{reply-to-comments}}

C1: Here is the punch line: I believe that conclusions of the manuscript
are flawed. Enclosed to this response is a counter example which
demonstrates this (in the pdf file).

R1: We would appreciate much better this explicit comment if this
Referee had raised it in the first report.

But let us respond it by restating first the stream of our work: I) In
high-quality MF states or quick MF dynamics (criteria have been given
around Eq.(7) ), the CPR of a MF Josephson junction is sin(theta/2) of
4pi period; for adiabatic MF dynamics, 2pi-period CPR with a sharp drop
at pi appears (see Eq. (9) and red dot line in Fig. 2(a)); II) In both
cases, a SQUID with the MF Josephson junction in one branch and a
conventional junction in the other branch exhibits a directionally
asymmetric critical current; III) This phenomenon, if found
experimentally, is an important signal for MF states, since an
asymmetric critical current in SQUID in weak current regime has not been
found experimentally so far for other reasons.

This Referee raised a sawtooth-shaped CPR with 2pi period, and discussed
that it can also induce an asymmetric critical current in SQUID. A
sawtooth-shaped CPR was known theoretically in ballistic systems, such
as SNS junction in clean limit, for long time. However, to the best of
our knowledge, there is no experimental report on a phenomenon
originated from this property. This indicates that the condition of
ballistic transport in conventional junction is hardly satisfied in
reality.

In contrary, the MF state is a topological feature of the
superconductor/ spin-orbit coupling semiconductor heterostructure under
appropriate magnetic field, and is robust to local perturbations. This
provides a good chance to detect the asymmetric critical current in the
SQUID proposed in our work. In this sense, our proposal is physically
novel and concrete.

Therefore, although there may be other possible mechanisms to cause
asymmetric critical current of SQUID on a pure theoretical level, with a
general anti-symmetry breaking documented in our manuscript, they are
not plausible in real systems, and cannot serve as counter example to
our proposal.

Moreover, we suggest that one can rotate the Zeeman field for realizing
the MF states, a technique adopted by Kouwenhoven et al., and transform
the topological state with MFs to a trivial state without MF. This
double check on the appearance and disappearance of asymmetric critical
current enhances greatly the uniqueness of MF states as its origin.

C2: Intentionally or not, in their manuscript and response, the authors
use sleight of hand. Typical example of this is their answer to my
comment 3 from the previous round of reviews which was: How come both
2pi and 4pi periodic CPRs show asymmetry in Ic if such asymmetry is
(``an unambiguous'') signature for MFs? The authors start their answer
with a definitive ``There is no contradiction here'' and just a few
lines below admit that ``\ldots{} a complex wave-shape of 2pi appears
which also leads to the asymmetry of critical current'' (``complex''
here means complicated).

R2: As mentioned in R1, the sin(theta/2) CPR with 4pi period and
2pi-periodic CPR with the sharp drop at pi (in Eq.(9) and red dot line
in Fig. 2a) are derived for the two limits of MF dynamics. Both of them
induce asymmetric critical current, and thus both serve as evidence for
MFs.

During the review process, Referee A suggested us to make a more concise
introduction, in order to improve the accessibility of our work to broad
audience. We then included only the 4pi CPR in high-quality MF system in
the introductory part in later versions of our manuscript.

In order to avoid possible misunderstanding, as pointed out by this
Referee in the last and present report, we re-phrase the introductory
part into ``unconventional current-phase relation (CPR) in MF states,
such as \(\sin (\theta/2)\) in a high-quality MF junction, and the
conventional one \(\sin \theta\)''. We hope that this introduction, plus
detailed discussions presented in the main body of text, will not leave
misunderstanding to a careful reader.

We wish to put it clear that our main statement has never been changed
since our first submission, and there is no ``sleight of hand'' in our
reply and manuscript.

C3: I take this as the coup de grace: This statement invalidates the
main claim of the manuscript since any practical (experimental) weak
link, unlike theoretical models, has a complex CPR which could be quite
different from a regular sine wave. Yes, MFs may produce the field
asymmetry in the critical current but so do billion other things! The
authors rescue boat - the tunability of MFs with the Zeeman coupling -
is of no help here.

R3: We have to point out that the asymmetric critical current in SQUID
is not a common phenomenon in an experimental point of view, in contrary
to the belief of this Referee. As a matter of fact, the known case for
asymmetric critical current is caused by the magnetic field self-induced
by strong current, which is however not the regime under concern in the
context of MF states.

It is true that, in addition to the leading term of sine function of
phase, Josephson current may have corrections of higher harmonics due to
many reasons. However, for weak current under concern in the present
context, correction terms are suppressed quickly, leaving a pure
sin(theta) CPR. It is the reason why traditional SQUIDs exhibit
symmetric critical current in experiments. In a sharp contrast, a MF
Josephson junction exhibits either a sin(theta/2) CPR of 4pi period or
2pi-periodic CPR with complicated shape even for small critical current.

Finally, we wish to say that tunability of MFs with Zeeman coupling is
not a ``rescue boat'', it is a very important step to check MF state, as
realized by Kouwenhoven et al.~As a matter of fact, many recent
experiments use this technique, making it common in this field. This
process can rule out other possible unknown mechanisms, and strongly
enhance the significance of the asymmetric critical current as an
evidence for MFs.

C4: The authors also want their work to look ``novel'', as if it appears
in vacuum, without mentioning previous conceptually very similar
proposals to detect MF. In fact, one of the such proposals has been
realized (more than half a year ago!) and which the authors ignore is
reported here: 1204.4212.

R4: We would appreciate much better this comment if this Referee had
raised it in the first report. As for the work of 1204.4212, we wish to
remind this Referee that the present work was actually submitted to PRL
half year ago, almost the same time when the work 1204.4212 was uploaded
to arXiv. It is therefore not surprising that it did not appear in our
reference list. We add this work as a note added to proof.

Summary of changes: 1) We summarize the main point in reply to this
Referee into a section of `Discussion' at the end of manuscript, and
remove the `Conclusion' section. Meanwhile, we delete the last part of
the section of `Asymmetric critical current', which discussed the same
point as in Discussion.

2)In the third paragraph of Introduction we rephrase the brief
description of our work as `We predict a directionally asymmetric
critical current in this SQUID due to the interplay between the
unconventional current-phase relation (CPR) in MF states, unconventional
current-phase relation (CPR) in MF states, such as \(\sin (\theta/2)\)
in a high-quality MF junction, and the conventional one
\(\sin \theta\)'.

3)In the section of `Rectification effect' section, we delete the last
part in order to save space.

4)We include a `Note Added in Proof' at the end of the manuscript, and
refer to the work 1204.4212.

5)In the list of references, we add J. A. van Dam, Yu.V. Nazarov, E. P.
A. M. Bakkers, S. De Franceschi, and L. P. Kouwenhoven, Nature (London)
\textbf{442}, 667 (2006). (Ref.\textasciitilde{}34 in the revised
manuscript).

\hypertarget{remarks-intended-solely-for-the-editor-1}{%
\subsubsection{Remarks intended solely for the
editor:}\label{remarks-intended-solely-for-the-editor-1}}

Dear Dr.~Donavan Hall Senior Assistant Editor

Thank you for the communication on our manuscript LR13604 on Nov.~22. We
have read carefully the second report from Referee C. Unfortunately we
find this report is unfair in the point of view of reviewing process,
and meanwhile not valuable physically. Referee C introduced a detailed
model as a counter example to our work in the second report, which s/he
could have done in the first report. S/he criticized that we did not
refer to previous works `intentionally', a comment s/he could have
raised in the first report too. This made the reviewing process
unnecessarily delayed, and reduced much, if not deprived completely, our
opportunity to rebut and to improve the manuscript appropriately. In
what follows we reveal briefly but clearly that his/her comments are not
valuable scientifically. This Referee introduced a sawtooth-shaped CPR
with 2pi period, and discussed that it can also induce an asymmetric
critical current in SQUID. A sawtooth-shaped CPR was known theoretically
in ballistic systems, such as clean SNS junction, for long time.
However, there is no experimental report on a phenomenon related to this
property in literature, which indicates that the condition of ballistic
transport in conventional junction is hardly satisfied in reality. In
contrary, the MF state is a topological feature of the superconductor/
spin-orbit coupling semiconductor heterostructure under appropriate
magnetic field, and is robust to local perturbations. This provides a
good chance to detect the asymmetric critical current in the SQUID
proposed in our work. Therefore, our proposal is physically novel and
concrete. In short, although there may be other possible mechanisms to
cause asymmetric critical current of SQUID on a purely theoretical
level, they are not plausible in real systems, and cannot serve as a
counter example to our proposal. S/he criticized that a work in
1204.4212 was dropped from our reference list `intentionally'. This is
clearly not the case since our work and 1204.4212 were submitted or
uploaded to arXiv at almost the same time, a half year ago. Therefore,
we have to say that the opinion of this Referee is biased, and some of
the comments are even emotional, and can hardly be taken as a base for
academic judgment on our work. In order to avoid unnecessary
misunderstanding, we improve our manuscript. For details please see the
Summary of changes and the revised manuscript. We wonder whether Referee
A can be consulted again, who made very sharp and valuable comments on
our work, and is knowledgeable. We would like to thank you in advance
for your reconsideration.

Sincerely yours, Xiao Hu on behalf of all authors

    \hypertarget{manipulating-edge-majorana-fermions-in-the-vortex-state-of-weakly-coupled-s-wave-superconductorsemiconductorferromagnet-heterostructures}{%
\section{Manipulating edge Majorana fermions in the vortex state of
weakly coupled s-wave superconductor/semiconductor/ferromagnet
heterostructures}\label{manipulating-edge-majorana-fermions-in-the-vortex-state-of-weakly-coupled-s-wave-superconductorsemiconductorferromagnet-heterostructures}}

\hypertarget{prl-round-1}{%
\subsection{PRL Round 1}\label{prl-round-1}}

\hypertarget{report-of-referee-a-lp12629liang}{%
\subsubsection{Report of Referee A --
LP12629/Liang}\label{report-of-referee-a-lp12629liang}}

In the present manuscript, the authors consider Majorana fermions (MFs)
occurring in an s-wave superconductor-semiconductor with S-O coupling
-ferromagnetic insulator (S/SM/FI) heterostructure (brick) by assuming
the presence of a single vortex in each brick and argue that their
computation implies the nonAbelian statistics of the MFs. They stress
that they have studied the edge MF rather than that at the vortex core
which have been studied by others so far.

The preceding studies addressing the non-Abelian statistics of MFs as
the statistics obeyed by the vortices have been of interest through
comparison with the previous study of that of vortices in a bose liquid
as an issue in quantum condensed matter physics. This study is
interesting for experts in that the non-Abelian statistics has been
verified through an explicit computation. However, it has already been
obvious as a theoretical result that the MFs obey the non-Abelian
statistics. So, I do not feel that the present study is innovative.
Actually, the present study is never based on a new idea.

In conclusion, this study appears to be suitable for a scientific
journal that treats the topological order. However, it is not
appropriate for PRL which focuses on innovative works.

\hypertarget{reply-to-report-of-referee-a-lp12629liang}{%
\subsubsection{Reply to Report of Referee A --
LP12629/Liang}\label{reply-to-report-of-referee-a-lp12629liang}}

\begin{quote}
Comment 1: In the present manuscript, the authors consider Majorana
fermions (MFs) occurring in an s-wave superconductor-semiconductor with
S-O coupling -ferromagnetic insulator (S/SM/FI) heterostructure (brick)
by assuming the presence of a single vortex in each brick and argue that
their computation implies the nonAbelian statistics of the MFs. They
stress that they have studied the edge MF rather than that at the vortex
core which have been studied by others so far.
\end{quote}

\begin{quote}
The preceding studies addressing the non-Abelian statistics of MFs as
the statistics obeyed by the vortices have been of interest through
comparison with the previous study of that of vortices in a bose liquid
as an issue in quantum condensed matter physics.
\end{quote}

Reply 1: The most important result of our analyses, both numerical and
analytical, is to indicate unambiguously that edge MFs can be
transported in an easy way, namely just applying gate voltage at
point-like constriction junction between samples, based on the intrinsic
topological property of MFs. Non-Abelian statistics is not our main,
innovative contribution, and conceptually it is not different from
previous works on core MFs in vortex state and end MFs in nanowires. We
check the braiding of edge MFs carefully in our manuscript, because it
is crucially important for any implementation of TQC, and not trivial in
the present case.

\begin{quote}
Comment 2: This study is interesting for experts in that the non-Abelian
statistics has been verified through an explicit computation. However,
it has already been obvious as a theoretical result that the MFs obey
the non-Abelian statistics. So, I do not feel that the present study is
innovative. Actually, the present study is never based on a new idea.
\end{quote}

Reply 1: We cannot agree with the comment from this Referee.

It was shown by Read and Green (PRB 2000) that a vortex in spinless
chiral p-wave superconductor accommodates a MF as the zero-energy
excitation. Ivanov (PRL 2001) then showed that braiding vortices rotates
the wave functions in the ground state degenerate subspace following the
non-Abelian statistics, and thus the vortex state can be used to
implement topological quantum computation (TQC). However, there are two
problems in practice with this wonderful idea, namely (1) spinless
chiral p-wave superconductivity has not been confirmed clearly in any
compounds, and (2) transporting vortices is by no means an easy task.
Very recently, it has been revealed by Lutchyn et al. (PRL 2010) that a
S/SM/FI hetero structure, with S for s-wave superconductor, SM for
spin-orbit coupling semiconductor (SM), and FI for ferromagnetic
insulator (FI), works similarly to the spinless chiral p-wave
superconductor, which solves the first problem.

In the present work, we propose a way to solve the second problem by
using the edge MFs, which are the counterpart of MFs at vortex cores and
appear at the sample edge as guaranteed by the particle-hole symmetry.
Our device is composed by samples of S/SM/FI hetero structure, each
trapping one vortex at the center and connected to others with
point-like constriction junctions, which we called a brick. Taking the
advantage of their distributions at brick edges, we demonstrate that
edge MFs can be transported and exchanged simply by turning on and off
gate voltages at the point-like constriction junctions in a designed
sequence. It should be emphasized that the transportation of edge MFs in
our system has its root in the topological property, namely edge MF
states appear when the sample perimeter (i.e.~edge) includes odd
vorticity, and should disappear for even vorticity. With our design, one
can circumvent the bottleneck of vortex manipulation in Ivanov's
original idea. Our present proposal therefore bridges successfully the
fundamental topological features of MFs and practical devices for TQC.
Our system has advantages as compared with the existing proposals for
manipulating MFs in literatures. In the proposal by Fu et al. (PRL 2008)
based on superconductor and topological insulator, manipulation of MFs
is based on tuning the phase of superconductivity, which is not a
trivial task; additionally the non-Abelian property was not proved for
their setup. In the proposal by Alicea et al. (Nat. Phys. 2010) based on
network of nanowires of spin-orbit coupling 1D semiconductor under
magnetic field in proximity to an s-wave superconductor, the non-Abelian
feature has been addressed. However, the end MFs in their setup should
be moved by application of gate voltage over the whole 1D nanowires, a
process which should be very delicate in order to avoid destroying the
whole topological state. In contrast, in our present setup, one has only
to tune gate voltages at point-like constriction junctions, which drives
easily the edge MFs to desired positions while keeping the topological
property of the whole system.

In short, our proposal reveals for the first time that the topological
feature of edge MFs can be used for stable and easily-implemented MF
braidings, which is the most important step of TQC, and thus should be
considered as innovative.

The non-Abelian statistics for MFs is not proven for the first time, as
pointed out by this Referee, and actually mentioned explicitly in the
introduction of manuscript. However, since non-Abelian statistics is
crucially important for TQC, it should be reconfirmed in any system
proposed as its implementation.

Additionally, we wish to bring the attention of this Referee to the
point that the proof of non-Abelian statistics of core MFs by Ivanov
cannot be applied directly to the edge MFs in the present setup, since
it is not trivial to indentify the motion of edge MFs around a vortex
during the braiding. We also wish to notice that Alicea et al recently
published an important paper (Nat. Phys. 2010) to prove in a detailed
way the non-Abelian feature of end MFs in nanowires, although it has
already been known since the work by Kitaev (Ann. Phys. 2003).

\begin{quote}
Comment 2: In conclusion, this study appears to be suitable for a
scientific journal that treats the topological order. However, it is not
appropriate for PRL which focuses on innovative works.
\end{quote}

Reply 2: As discussed in the above reply, our work contains new results,
which, on one hand, have their root in the topological property of the
system, and, on the other hand, exhibit potential as an implementation
of TQC. Therefore, we believe the present work will receive much
attention from the broad scope of readers of PRL. Reconsideration of
this Referee is highly appreciated.


    % Add a bibliography block to the postdoc
    
    
    
    \end{document}
